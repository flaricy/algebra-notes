%----------------------------------------------------------------------------------------
%	PACKAGES AND OTHER DOCUMENT CONFIGURATIONS
%----------------------------------------------------------------------------------------

\documentclass[11pt,fleqn]{book} % Default font size and left-justified equations

\usepackage[top=3cm,bottom=3cm,left=3.2cm,right=3.2cm,headsep=10pt,letterpaper]{geometry} % Page margins
\usepackage{CJKutf8}
\usepackage{xcolor} % Required for specifying colors by name
\definecolor{ocre}{RGB}{52,177,201} % Define the orange color used for highlighting throughout the book

% Font Settings
\usepackage{avant} % Use the Avantgarde font for headings
%\usepackage{times} % Use the Times font for headings
\usepackage{mathptmx} % Use the Adobe Times Roman as the default text font together with math symbols from the Sym­bol, Chancery and Com­puter Modern fonts
\usepackage{microtype} % Slightly tweak font spacing for aesthetics
\usepackage[utf8]{inputenc} % Required for including letters with accents
\usepackage[T1]{fontenc} % Use 8-bit encoding that has 256 glyphs
\usepackage{amsthm}
\usepackage{quiver} % to draw commutative diagrams

% Bibliography
\usepackage[style=alphabetic,sorting=nyt,sortcites=true,autopunct=true,babel=hyphen,hyperref=true,abbreviate=false,backref=true,backend=biber]{biblatex}
\addbibresource{bibliography.bib} % BibTeX bibliography file
\defbibheading{bibempty}{}

\input{structure} % Insert the commands.tex file which contains the majority of the structure behind the template

%----------------------------------------------------------------------------------------
%	Definitions of new commands
%----------------------------------------------------------------------------------------

\def\R{\mathbb{R}}
\newcommand{\cvx}{convex}
\begin{document}
\begin{CJK}{UTF8}{gkai} % gbsn 宋体
%----------------------------------------------------------------------------------------
%	TITLE PAGE
%----------------------------------------------------------------------------------------

\begingroup
\thispagestyle{empty}
\AddToShipoutPicture*{\put(0,0){\includegraphics[scale=1.25]{esahubble}}} % Image background
\centering
\vspace*{5cm}
\par\normalfont\fontsize{35}{35}\sffamily\selectfont
\textbf{Algebra (Honor Track) Spring 2024}\\
\vspace*{0.4cm}
{\Huge \textbf{Communitative Algebra}}\par % Book title
\vspace*{0.4cm}
{\Huge Notes}\par % Author name
{\Large ymy}\par
\endgroup

%----------------------------------------------------------------------------------------
%	COPYRIGHT PAGE
%----------------------------------------------------------------------------------------

\newpage
~\vfill
\thispagestyle{empty}

%\noindent Copyright \copyright 2014 Andrea Hidalgo\\ % Copyright notice

\noindent \textsc{Personal use}\\

\noindent {https://github.com/flaricy/algebra-notes}\\ % URL

\noindent The author hopes to take notes while learning abstract algebra. Reference books are \textit{Introduction to communitative algebra} by \textit{Atiyah, Michael}. Starts from Feb 21st, 2024. 

%----------------------------------------------------------------------------------------
%	TABLE OF CONTENTS
%----------------------------------------------------------------------------------------

\chapterimage{head1.png} % Table of contents heading image

\pagestyle{empty} % No headers

\tableofcontents % Print the table of contents itself

%\cleardoublepage % Forces the first chapter to start on an odd page so it's on the right

\pagestyle{fancy} % Print headers again

\newpage
\thispagestyle{empty}
\centering 
\vspace*{10cm}
\textit{This page is intentionally left blank.}
%----------------------------------------------------------------------------------------
%	CHAPTER 1
%----------------------------------------------------------------------------------------

\chapterimage{head2.png} % Chapter heading image
\chapter{Group Theory}
\section{Groups and subgroups}
\begin{definition}
	[direct product] Let $(G, *)$ and $(H, \circ )$ be groups, then we may form a new group structure
	on $G \times H$ with group operation given by 
	\[(g, h) \star (g', h') = (g*g', h\circ h') \]
	This is called the {\bf direct product} of G and H.
\end{definition}

\subsection{Important Examples of Groups} 
\begin{definition}
	[Dihedral groups 二面体群] \[D_{2n} = \text{symmetric group of a reguler n-gon}\]
	It can be rewritten as 
	\[D_{2n} = \langle r,s | r^n = 1, s^2 = 1, rsr = s^{-1}\rangle \]
\end{definition}

\begin{definition}
	[Permutation Groups] Let $\Omega$ be a set. The set 
	\[S_\Omega = \{\text{bijections } \sigma: \Omega \xrightarrow{\thicksim} \Omega\}\]
	admits a group structure:
	\begin{itemize}
		\item the group operation is composition
		\item the identity element is $id$
		\item the inverse of the element $\sigma$ is the inverse map.
	\end{itemize}
	This $S_\Omega$ is called the symmetry group or the permutation group of $\Omega$.
	When $\Omega = \{1,2 ,...,n\}$, we write $S_n$ instead.
\end{definition}

\begin{definition}
	[cyclic groups] A group $H$ is called cyclic if it can be generated by one element $x$, i.e.
	\[H = \langle x\rangle \]	
\end{definition}

\begin{lemma}
	There are 2 kinds of cyclic groups up to isomorphism. \\
	(1) $H \cong \textbf{Z}_n$ \\
	(2) $H \cong \textbf{Z}$
\end{lemma}

\begin{definition}
	[The quaternion group] 
	\[Q_8 = \{1,-1,i,-i,j,-j,k,-k\}\]
\end{definition}

\section{cosets, Lagrange theorem, quotient groups}
\subsection{Conjugation, normal subgroups, and quotient groups.}
\begin{definition}
	[conjugate] Let $a, g \in G$, then $gag^{-1}$ is called the {\bf conjugate of $a$ by $g$}. 
\end{definition}
\begin{definition}
	[定义-命题] If $H$ is a subgroup of $G$ and $g \in G$, then $gHg^{-1}$
	:= $\{ghg^{-1}
	| h \in H\}$ is a
	subgroup, called the conjugate of H by g
\end{definition}
\begin{proof}
	We just need to verify that $\forall a,b \in H$, $gag^{-1}\cdot(gbg^{-1})^{-1} \in gHg^{-1}$. 
\end{proof}

\begin{definition}
	[normal subgroup] If $H \leq G$ and all conjugates of $H$ is $H$ itself, we denote $H \unlhd G$.
	Note that this condition is also equivalent to $gH = Hg$
	(as subsets) for any $g \in G$. 
\end{definition}

\begin{definition}
	[quotient group] Let $H \unlhd G$, then $\forall a,b \in G$, we define
	\[aH \cdot bH:=\{kl | k\in aH, l \in bH\} = abH\]
	as subsets of $G$. This defines a group structure on $G/H$, called the {\bf quotient group} or the {\bf factor group} of $G$ by $H$.  
\end{definition}

\subsection{Some Technical Results}
\begin{proposition}
	Let $H$ and $K$ be subgroups of a group $G$. Define $HK = \{hk | h\in H, k\in K\}$. When $G$ is finite, we have 
	\[|HK| = \frac {|H| \cdot |K|} {|H \cap K|}\]
\end{proposition}

\begin{proof}
	{\it to be written}
\end{proof}

The following lemmas tells when $HK$ is a (normal) subgroup.
\begin{lemma}
	Let $H$ and $K$ be subgroups of $G$. If $HK = KH$ as sets, then $HK$ is a subgroup of $G$.
	In particular, if $K$ is a normal subgroup, then $hK = Kh$ for any $h \in H$, and thus $HK = KH$ is a subgroup of $G$.
\end{lemma}
\begin{proof}
	We need to verify that $\forall h_1k_1 \cdot (h_2k_2)^{-1} = h_1k_1k_2^{-1}h_2^{-1}\in HK$. Since $h_1(k_1k_2^{-1}) \in HK = KH$, there exists $h,k$ such that $h_1k_1k_2^{-1} = kh$. 
	Then $khh_2^{-1} \in KH = HK$.
\end{proof}

\begin{lemma}
	If $H, K$ are both normal subgroups of $G$, then $HK$ is also a normal subgroup of $G$.
\end{lemma}
\begin{proof}
	$\forall g \in G$, we have $gHK = HgK = HKg$. 	
\end{proof}

\subsection{homomorphism}
\begin{definition}
	[Kernel as a group homomorphism] For a homomorphism $\phi: G \to H$ of groups, the {\bf kernel} is 
	\[ker \;\phi = \{g \in G | \phi(g) = e_H\}\]
\end{definition}

\begin{lemma}
	Let $\phi:G\to H$ be a group homomorphism. \\
	(1) The image $\phi(G)$ is a subgroup of $H$. \\
	(2) The kernel $ker \; \phi$ is a normal subgroup of $G$.
\end{lemma}
\begin{proof}
	(1) It follows from that $\phi(g_1)\phi(g_2)^{-1} = \phi(g_1g_2^{-1}) \in \phi(G)$ \\
	(2) If $g_1, g_2 \in ker \; \phi$, then 
	\[\phi(g_1g_2^{-1}) = e_He_H^{-1} = e_H\]
	For any $g' \in G$, and any $g \in ker \ \phi$, \[\phi(g'gg'^{-1}) = \phi(g')e_H\phi(g')^{-1} = e_H\]  
\end{proof}

\begin{lemma}
	A homomorphism $\phi: G \to H$ of groups is injective if and only if $ker \; \phi = \{e_G\}$.
\end{lemma}

\section{isomorphism theorems, composition series, statement of Holder Theorem}
\subsection{isomorphism theorems}

\begin{theorem}
	[The first isomorphism theorem] If $\phi: G \to H$ is a homomorphism of groups, then $ker \; \phi \unlhd G$ and \[
		G / ker \phi \cong \phi(G) \]
\end{theorem}

\begin{theorem}
	[The second homomorphism theorem] Let $G$ be a group, and let $A \leq G$ be a subgroup and $B \unlhd G$ a normal subgroup. Then $AB$ is a subgroup of G, $B \unlhd AB, A\cap B \unlhd A$, and 
	\[AB/B \cong A/(A\cap B)\]
\end{theorem}
\begin{proof}
	By lemma 1.2.2 we know $AB$ is a subgroup of $G$. \\
	For any $ab \in AB$, since $B$ is normal to $G$, $abB = aB = Ba$ and $aB = aBb = Bab$. So $B \unlhd AB$. \\
	It is clear that $A\cap B \leq A$. For any $a \in A, x \in A\cap B$, we have $axa^{-1} \in B$, since $B$ is normal. Also $axa^{-1} \in A$, since $x \in A$. So $A \cap B \unlhd A$. \\
	To show the isomorphism, we define $\phi: AB \to A/(A\cap B)$ by $\phi(ab) = a(A\cap B)$. It's easy to verify that $\phi$ is well-defined, surjective and a homomorphism, with $ker \phi = B$. By Theorem 1.3.1, we know the statement is true.
	\[\begin{tikzcd}
		AB && {A/(A\cap B)} \\
		\\
		& {AB/B}
		\arrow["\phi", two heads, from=1-1, to=1-3]
		\arrow["q"', from=1-1, to=3-2]
		\arrow["f"', from=3-2, to=1-3]
	\end{tikzcd}\]
\end{proof}

\begin{theorem}
	[The third isomorphism theorem] Let $G$ be a group and $H, K$ be normal subgroups with $H \leq K$. Then $K/H \unlhd G/H$, and 
	\[(G/H)/(K/H) \cong G/K\] 
\end{theorem}
\begin{proof}
	Consider the map 
	\[\begin{tikzcd}
		{\phi:} & {G/H} && {G/K} \\
		& gH && gK
		\arrow[from=1-2, to=1-4]
		\arrow[maps to, from=2-2, to=2-4]
	\end{tikzcd}\]
	\begin{itemize}
		\item $\phi$ is well-defined. We can simply redefine $\phi$ as $\phi(gH) = gH\cdot K = gK$ as product of subsets of $G$.
		\item $\phi$ is homomorphism. Easy to verify.
		\item $\phi$ is surjective.
		\item ker $\phi = \{gH | gK = K\} = \{gH | g \in K\} = K/H$. So $K/H \unlhd G/H$. And by the first isomorphism theorem, we statement holds.  
	\end{itemize}
\end{proof}

\begin{theorem}
	[The fourth isomorphism theorem/ Lattice isomorphism theorem] Let $G$ be a group and $N \unlhd G$. Then there is a bijection
	\[\begin{tikzcd}
		{\{subgroups \ of \ G \ containing \ N\}} && {\{subgroups \ of \ G/N\}} \\
		A && {A/N} \\
		{\pi^{-1}(\overline A)} && {\overline{A}}
		\arrow[tail reversed, from=1-1, to=1-3]
		\arrow[maps to, from=2-1, to=2-3]
		\arrow[maps to, from=3-3, to=3-1]
	\end{tikzcd}\]
	where $\pi: G \to G/N$ is the natural projection. \\
	This bijection preserves
	\begin{itemize}
		\item inclusion of groups 
		\item intersections
		\item normality of subgroups 
		\item quotients of subgroups 
	\end{itemize}
	Visually, we have: Lattice of subgroups of G containing N $\iff$ Lattice of subgroups of G/N. 
\end{theorem}

\section{Lattice}
\begin{definition}
	Let (S, $\leq$) be a set equipped with a partial order. (S, $\leq$) is called a {\it Lattice} if any $x, y \in S, \{x, y\}$ has a maximal lower bound and a minimal upper bound.  
	The lower bound is denoted by $x \wedge y$, while the upper bound is denoted by $x \vee y$.
\end{definition}

\begin{example}
	设 n 为正整数,$A_n$ 为 n 的所有正因数的几何,则 $A_n$ 关于整除关系构成格。
\end{example}
\begin{example}
	设 $P(B)$ 为 $B$ 的幂集,则 $P(B)$ 关于包含关系 $\subseteq$ 构成格,称为幂集格.
\end{example}

\begin{example}
	[子群格] 群G的所有子群,关于包含关系。
\end{example}

\section{composition series, Jordan-Holder Theorem, simplicity of An, direct product groups}
\begin{definition}
	[composition series] In a group $G$, a series of subgroups 
	\[\{0\} = N_0 \leq N_1 \leq ... \leq N_k = G\]	
	such that $N_{i-1} \unlhd N_i$ and $N_i / N_{i-1}$ is a simple group for $1 \leq i \leq k$ is called {\bf composition series}. In this case, $N_i / N_{i-1}$ is called a {\bf composition factor}.
\end{definition}

\begin{definition}
	[solvable] A group G is called {\bf solvable} if there exists a composition series 
	\[\{0\} = N_0 \leq N_1 \leq ... \leq N_k = G\] 
	such that $N_i / N_{i-1}$ is abelian.
\end{definition}

\begin{corollary}
	a finite group is solvable if and only if all the composition factors are $\mathbf{Z}_p$.
\end{corollary}

\begin{theorem}
	[Jordan-Holder] Let G be a non-trivial group, \\ 
	(1) G has a composition series. \\(2) Assume that a group $G$ has the following two composition series, 
	\[\{0\} = A_0 \leq A_1 \leq ... \leq A_m = G, \quad \{0\} = B_0 \leq B_1 \leq ... \leq B_n = G\]
	then $m = n$ and there exists a bijection $\sigma:\{1,2,...,m\}\to \{1,2,...,n\}$
	\[A_{\sigma(i)} / A_{\sigma(i-1)} \cong B_i / B_{i-1}\]
	for $i = 1, 2, ... m$
\end{theorem}

\begin{proof}
	{\it to be written}
\end{proof}

\subsection{The simplicity of An, n >= 5}

\begin{proposition}
	
\end{proposition}
%%% -------------------------
%%%  chapter{Rings and Ideals}
%%%
\chapterimage{destiny2_resized.jpg} % Chapter heading image
\chapter{Rings and Ideals}
If not pointed out specifically, the notion "ring" refers to a commutative ring with an identity element. 

\section{rings, ideals, quotient rings}
\begin{definition}
	[ring homomorphism] Let $A, B$ be rings, $f : A\to B$ is a homomorphism when \\
	(1) $f(x+y) = f(x) + f(y)$. So $f$ is a homomorphism of abelian groups. \\
	(2) $f(xy) = f(x)f(y)$, $f(1) = 1$. So $f$ is a homomorphism between the monoids $(A, \cdot)$ and $(B, \cdot)$.

\end{definition}
\begin{definition}
	[ideal of a ring] An ideal $I$ of a ring $A$ is an additive subgroup and is such that $A\I \subseteq I$.
\end{definition}
\begin{example}
	Every ring $A$ has 2 trivial ideals: $\{0\}$ and $A$.
\end{example}

Below, $I$ denotes the ideal of ring $A$.
\begin{definition}
	[quotient ring] Define multiplication in the quotient group $A/\I$ by\\
	\[(a + I) \cdot (b + I) = ab + I\]
	It is well defined. Now $A/I$ is made into a ring called the {\it quotient ring}. The mapping $\phi: A \to A/I$ which maps each $x \in A$ to its coset $x + I$ is a surjective ring homomorphism.
\end{definition}

\begin{proposition}
	There is a one-to-one order preserving correspondence between 
	\[\begin{tikzcd}
		{\{J|I\subseteq J \subseteq A, J:ideal\}} && {\{\overline{J} | ideal \; \overline{J} \subseteq A/I\}} \\
		J && {J + I} \\
		{\phi^{-1}(\overline{J})} && {\overline{J}}
		\arrow["{1:1}", tail reversed, from=1-1, to=1-3]
		\arrow[maps to, from=2-1, to=2-3]
		\arrow[maps to, from=3-3, to=3-1]
	\end{tikzcd}\]
\end{proposition}
\begin{proof}
	First, Let's show that $J + I$ is an ideal in $A/I$. \\
	$J + I$ is abelian : trivial; $\forall x + I \in A/I, (x+I)\cdot(J+I) = (Jx + I) \subseteq (J+I) $, since $J$ is an ideal. \\
	Second, we can verify this mapping to be invertible.

\end{proof}

\begin{corollary}
	If $f: A\to B$ is any ring homomorphism, the {\it kernel} of $f(=f^{-1}(0))$ is an ideal of $A$, and the image of $f (= f(A))$ is a subring $C$ of $B$, but may not be an ideal.
\end{corollary}
\begin{proof}
	Consider the embedding mapping\[\begin{tikzcd}
		{\mathbb{Q} } && {\mathbb{Q}[X]}
		\arrow[hook, from=1-1, to=1-3]
	\end{tikzcd}\]
	The image is absolutely not an ideal.
\end{proof}

\begin{theorem}
	[fundamental homomorphism theorem] $f: A \to B$ is a ring homomorphism, $I$ is the kernel of $f$, $g(a + I) := f(a)$ then $g$ is a ring isomorphism.
	\[\begin{tikzcd}
		A && {Im(f)} && B \\
		\\
		&& {A/I}
		\arrow["f", two heads, from=1-1, to=1-3]
		\arrow[hook, from=1-3, to=1-5]
		\arrow["g", from=3-3, to=1-3]
		\arrow["\phi", from=1-1, to=3-3]
	\end{tikzcd}\]

\end{theorem}

\section{zero-divisors, nilpotent elements, units}
\begin{definition}
	[zero-divisor] a zero-divisor in a ring $A$ is an element $x$ for which there exists $y \neq 0$ in $A$ such that $xy = 0$
\end{definition}
\begin{definition}
	[integral domain] a ring with no zero-divisors $\neq 0$ and not a zero ring.
\end{definition}

\begin{definition}
	[nilpotent] An element $x \in A$ is {\it nilpotent} if $x^n = 0$ for some $n > 0$.
\end{definition}
\begin{remark}
	A nilpotent element is a zero-divisor.
\end{remark}

\begin{definition}
	[unit 可逆元] A unit in $A$ is an element $x$ such that $xy = 1$ for some $y \in A$. Note that $y$ is uniquely determined by $x$, and is written as $x^{-1}$. 
\end{definition}
\begin{remark}
	The units in $A$ form a abelian group under multiplication.
\end{remark}

\begin{definition}
	[field] A field is a ring $A$ which $1 \neq 0$ and every non-zero elem. is a unit.
\end{definition}

\begin{proposition}
Let A be a ring $\neq 0$. The following are equivalent:\\
(1) A is a field;\\
(2) The only ideals in A are ${0}$ and $(1)$; \\
(3) Every non-trivial homomorphism of A into a non-zero ring B is injective.	
\end{proposition}

\section{prime ideals and maximal ideals}
\newcommand{\fp}{\mathfrak{p}}
\newcommand{\fm}{$\mathfrak{m}$}
\begin{definition}
	[prime ideal]An ideal $\fp$ in $A$ is {\it prime} if $\fp \neq (1)$ and if $xy \in \fp \implies x\in \fp$ or $y \in \mathfrak{p}$ 
\end{definition}

\begin{definition}
	[maximal ideal] An ideal \fm in $A$ is {\it maximal} if \fm $\neq (1)$ and if there is no ideal $\alpha $ such that \fm $\subset \alpha \subset (1)$(strict inclusion).
\end{definition}
\begin{remark}
	\fm \text{ can} be $\{0\}$.
\end{remark}

\begin{proposition}
	$\fp$ is prime $\iff$ $A/\fp$ is an integral domain.
\end{proposition}
\begin{proof}
	Easy to verify.
\end{proof}

\begin{proposition}
	\fm \text{ }is maximal $\iff$ $A/m$ is a field. Hence, a maximal ideal is prime.
\end{proposition}
\begin{proof}
	By Proposition 2.1.1 and Proposition 2.2.1, the statement holds.
\end{proof}

\begin{proposition}
	If $f: A\to B$ is a ring homomorphism and $q$ is a prime ideal of $B$, then $f^{-1}(q)$ is a prime ideal in $A$.
\end{proposition}
\begin{proof}
	If $a, b \in A$ such that $f(a) = f(b) \in q$. Then $f(a-b) = f(a)-f(b) \in q$. Thus, $f^{-1}(q)$ is abelian. For any $a \in f^{-1}(q), x \in A$, we have 
	$f(ax) = f(a)f(x) \in Bq = q$. Thus, $f^{-1}(q)$ is an ideal. For any $a, b\in A, ab \in f^{-1}(q) \iff f(ab) \in q \iff f(a)\cdot f(b)\in q \iff f(a) \in q \vee f(b) \in q \iff a \in f^{-1}(q) \vee b \in f^{-1}(q) \iff f^{-1}(q)$ is a prime ideal.  	
\end{proof}

\begin{remark}
	If $m$ is a maximal ideal of $B$, it is not necessarily true that $f^{-1}(m)$ is maximal in $A$. Consider $A = \mathbb{Z}, B = \mathbb{Q}, m = \{0\}$.
\end{remark}

\begin{theorem}
	Every ring $A \neq 0$ has at least one maximal ideal.
\end{theorem}
This theorem relys on Zorn's Lemma. We first introduce it.

\begin{definition}
	[chain in a partially ordered set] Let $S$ be a non-empty partially ordered set. A subset $T$ of $S$ is a chain if either $x \leq y$ or $y \leq x$ for every pair of elements in $T$.
\end{definition}

\begin{lemma}
	[Zorn] If every chain $T$ of $S$ has an upper bound in $S$, then $S$ has at least one maximal element. Zorn's Lemma is equivalent to the axiom of choice.
\end{lemma}

\begin{proof}
	Let's prove theorem 2.3.4, using Zorn's Lemma. \\
	Let $\Sigma = \{I : I \ is \ ideal, I \neq (1)\}$. Order $\Sigma$ by inclusion. $\Sigma$ is not empty, since $0 \in \Sigma$. For each chain, consider the union as another ideal $\neq (1)$ to be an upper bound. Then Zorn's lemma yields that there is a maximal element.
\end{proof}
\begin{remark}
	If $A$ is Noetherian, we can avoid the use of Zorn's lemma.	
\end{remark}

\begin{corollary}
	If $a \neq (1)$ is an ideal of $A$, there exists a maximal ideal of $A$ containing $a$.
\end{corollary}
\begin{proof}
	Replace $\Sigma$ by $\{I: I \ is \ ideal \ containing \ a, I \neq (1)\}$ in the proof of Theorem 2.3.4 .
\end{proof}

\begin{corollary}
	Every non-unit of $A$ is contained in a maximal ideal.
\end{corollary}

\begin{definition}
	[local ring, residue field] If a ring $A$ has exactly one maximal ideal $m$ (e.g. fields), then $A$ is called a {\it local ring}. The field $k = A/m$ is called the residue field of $A$.
\end{definition}

\begin{proposition}
	Let $A$ be a ring and $m \neq (1)$ an ideal of $A$ such that $\forall x \in A - m$ is a unit in $A$. Then $A$ is a local ring and $m$ its maximal ideal.
\end{proposition}

First, we observe the following
\begin{lemma}
	Every element in a maximal ideal is not a unit.
\end{lemma}

\begin{proof}[proof of Proposition 2.3.8]
	From corollary 2.3.6 and lemma 2.3.9 we know $m$ is a maximal ideal. Also from lemma 2.3.9, we know there doesn't exists other maximal ideals. Thus, $A$ is a local ring.
\end{proof}

\begin{proposition}
	Let $A$ be a ring and $m$ a maximal ideal, such that every element of $1 + m$ is a unit in $A$. Then $A$ is a local ring.
\end{proposition}
\begin{proof}
	Make an analogy to Bezout Theorem. Let $x \in A - m$. Since $m$ is maximal, the ideal generated by $x$ and $m$ is $(1)$, hence there exists $y \in A, t\in m$ such that $xy + t = 1$.
	Thus $xy = 1 - t \in 1 + m$, which means $x$ is a unit. 
\end{proof}

\begin{example}
	$A = F[X_1,...,X_n], F$ : field. Let $f \in A$ be an irreducible polynomial. By unique factorization, the ideal $(f)$ is prime. When $n \geq 2$, it's not a {\it principal ideal domain}.	
\end{example}

\begin{example}
	Every ideal in $\mathbf{Z}$ is of the form $(m)$ for some $m \geq 0$. The ideal is prime $\iff$ $m = 0$ or is a prime number. For all ideals $(p)$ are maximal.
\end{example}

\begin{definition}
	[principal integral domain] an integral domain where every ideal is principal.
\end{definition}

\begin{proposition}
	Every non-zero prime ideal is maximal.
\end{proposition}
\begin{proof}
	[Hint] The cancellation law applies in the integral domain.
\end{proof}

\chapterimage{town_resized.png} % Chapter heading image
\chapter{Module Theory}

% file ending
\end{CJK}
\end{document}
