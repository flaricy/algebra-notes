%----------------------------------------------------------------------------------------
%	PACKAGES AND OTHER DOCUMENT CONFIGURATIONS
%----------------------------------------------------------------------------------------

\documentclass[11pt,fleqn]{book} % Default font size and left-justified equations

\usepackage[top=3cm,bottom=3cm,left=3.2cm,right=3.2cm,headsep=10pt,letterpaper]{geometry} % Page margins
\usepackage{CJKutf8}
\usepackage{xcolor} % Required for specifying colors by name
\definecolor{ocre}{RGB}{52,177,201} % Define the orange color used for highlighting throughout the book

% Font Settings
\usepackage{avant} % Use the Avantgarde font for headings
%\usepackage{times} % Use the Times font for headings
\usepackage{mathptmx} % Use the Adobe Times Roman as the default text font together with math symbols from the Sym­bol, Chancery and Com­puter Modern fonts
\usepackage{microtype} % Slightly tweak font spacing for aesthetics
\usepackage[utf8]{inputenc} % Required for including letters with accents
\usepackage[T1]{fontenc} % Use 8-bit encoding that has 256 glyphs
\usepackage{amsthm}
\usepackage{quiver} % to draw commutative diagrams

% Bibliography
\usepackage[style=alphabetic,sorting=nyt,sortcites=true,autopunct=true,babel=hyphen,hyperref=true,abbreviate=false,backref=true,backend=biber]{biblatex}
\addbibresource{bibliography.bib} % BibTeX bibliography file
\defbibheading{bibempty}{}

\input{structure} % Insert the commands.tex file which contains the majority of the structure behind the template

%----------------------------------------------------------------------------------------
%	Definitions of new commands
%----------------------------------------------------------------------------------------

\def\R{\mathbb{R}}
\newcommand{\cvx}{convex}
\begin{document}
\begin{CJK}{UTF8}{gkai} % gbsn 宋体
%----------------------------------------------------------------------------------------
%	TITLE PAGE
%----------------------------------------------------------------------------------------

\begingroup
\thispagestyle{empty}
\AddToShipoutPicture*{\put(0,0){\includegraphics[scale=1.25]{esahubble}}} % Image background
\centering
\vspace*{5cm}
\par\normalfont\fontsize{35}{35}\sffamily\selectfont
\textbf{Algebra (Honor Track) Spring 2024}\\
\vspace*{0.4cm}
{\Huge \textbf{Communitative Algebra}}\par % Book title
\vspace*{0.4cm}
{\Huge Notes}\par % Author name
{\Large ymy}\par
\endgroup

%----------------------------------------------------------------------------------------
%	COPYRIGHT PAGE
%----------------------------------------------------------------------------------------

\newpage
~\vfill
\thispagestyle{empty}

%\noindent Copyright \copyright 2014 Andrea Hidalgo\\ % Copyright notice

\noindent \textsc{Personal use}\\

\noindent {https://github.com/flaricy/algebra-notes}\\ % URL

\noindent The author hopes to take notes while learning abstract algebra. Reference books are \textit{Introduction to communitative algebra} by \textit{Atiyah, Michael}. Starts from Feb 21st, 2024. 

%----------------------------------------------------------------------------------------
%	TABLE OF CONTENTS
%----------------------------------------------------------------------------------------

\chapterimage{head1.png} % Table of contents heading image

\pagestyle{empty} % No headers

\tableofcontents % Print the table of contents itself

%\cleardoublepage % Forces the first chapter to start on an odd page so it's on the right

\pagestyle{fancy} % Print headers again

\newpage
\thispagestyle{empty}
\centering 
\vspace*{10cm}
\textit{This page is intentionally left blank.}
%----------------------------------------------------------------------------------------
%	CHAPTER 1
%----------------------------------------------------------------------------------------

\chapterimage{head2.png} % Chapter heading image
\chapter{Group Theory}
\section{Groups and subgroups}
\begin{definition}
	[direct product] Let $(G, *)$ and $(H, \circ )$ be groups, then we may form a new group structure
	on $G \times H$ with group operation given by 
	\[(g, h) \star (g', h') = (g*g', h\circ h') \]
	This is called the {\bf direct product} of G and H.
\end{definition}

\subsection{Important Examples of Groups} 
\begin{definition}
	[Dihedral groups 二面体群] \[D_{2n} = \text{symmetric group of a reguler n-gon}\]
	It can be rewritten as 
	\[D_{2n} = \langle r,s | r^n = 1, s^2 = 1, rsr = s^{-1}\rangle \]
\end{definition}

\begin{definition}
	[Permutation Groups] Let $\Omega$ be a set. The set 
	\[S_\Omega = \{\text{bijections } \sigma: \Omega \xrightarrow{\thicksim} \Omega\}\]
	admits a group structure:
	\begin{itemize}
		\item the group operation is composition
		\item the identity element is $id$
		\item the inverse of the element $\sigma$ is the inverse map.
	\end{itemize}
	This $S_\Omega$ is called the symmetry group or the permutation group of $\Omega$.
	When $\Omega = \{1,2 ,...,n\}$, we write $S_n$ instead.
\end{definition}

\begin{definition}
	[cyclic groups] A group $H$ is called cyclic if it can be generated by one element $x$, i.e.
	\[H = \langle x\rangle \]	
\end{definition}

\begin{lemma}
	There are 2 kinds of cyclic groups up to isomorphism. \\
	(1) $H \cong \textbf{Z}_n$ \\
	(2) $H \cong \textbf{Z}$
\end{lemma}

\begin{definition}
	[The quaternion group] 
	\[Q_8 = \{1,-1,i,-i,j,-j,k,-k\}\]
\end{definition}

\subsection{exercises}
\begin{example}
	suppose $G$ is cyclic. \\ 
	(1) Any subgroup of $G$ is cyclic. \\
	(2) If $|G| = \infty$, then all the subgroups but $\{e\}$ have order of infinity. \\
	(3) If $|G| = n$, then the order of subgroup is a factor of $n$. For every $d | n$, $G$ has only one $d$-ordered group.
\end{example}

\begin{example}
	$G$ is a group. $\forall x \in G$, $x^2 = 1$. Then $G$ is abelian.
\end{example}
\begin{remark}
	If $G$ has an element with order $\geq 3$, then there exists $a\neq b, a,b\neq 1$ such that $ab = ba$.
\end{remark}

\section{cosets, Lagrange theorem, quotient groups}
\subsection{Conjugation, normal subgroups, and quotient groups.}
\begin{definition}
	[conjugate] Let $a, g \in G$, then $gag^{-1}$ is called the {\bf conjugate of $a$ by $g$}. 
\end{definition}
\begin{definition}
	[定义-命题] If $H$ is a subgroup of $G$ and $g \in G$, then $gHg^{-1}$
	:= $\{ghg^{-1}
	| h \in H\}$ is a
	subgroup, called the conjugate of H by g
\end{definition}
\begin{proof}
	We just need to verify that $\forall a,b \in H$, $gag^{-1}\cdot(gbg^{-1})^{-1} \in gHg^{-1}$. 
\end{proof}

\begin{definition}
	[normal subgroup] If $H \leq G$ and all conjugates of $H$ is $H$ itself, we denote $H \unlhd G$.
	Note that this condition is also equivalent to $gH = Hg$
	(as subsets) for any $g \in G$. 
\end{definition}

\begin{definition}
	[quotient group] Let $H \unlhd G$, then $\forall a,b \in G$, we define
	\[aH \cdot bH:=\{kl | k\in aH, l \in bH\} = abH\]
	as subsets of $G$. This defines a group structure on $G/H$, called the {\bf quotient group} or the {\bf factor group} of $G$ by $H$.  
\end{definition}

\subsection{Some Technical Results}
\begin{proposition}
	Let $H$ and $K$ be subgroups of a group $G$. Define $HK = \{hk | h\in H, k\in K\}$. When $G$ is finite, we have 
	\[|HK| = \frac {|H| \cdot |K|} {|H \cap K|}\]
\end{proposition}

\begin{proof}
	Let $HK = \sqcup_{i=1}^n Hk_i$, where $k_i$ are representatives. Since $H\cap K \le K$, consider equivalence classes $K / (H \cap K)$. Define 
	\[\begin{tikzcd}
		{f:} & {\sqcup_{i=1}^n Hk_i} && {K/(H\cap K)} \\
		& {Hk_i} && {k_i(H\cap K)}
		\arrow[from=1-2, to=1-4]
		\arrow[maps to, from=2-2, to=2-4]
	\end{tikzcd}\]

	We can verify $f$ is well-defined, injective and surjective. So $|K| / |H\cap K| = n$ and the above proposition holds.
\end{proof}

The following lemmas tells when $HK$ is a (normal) subgroup.
\begin{lemma}
	Let $H$ and $K$ be subgroups of $G$. If $HK = KH$ as sets, then $HK$ is a subgroup of $G$.
	In particular, if $K$ is a normal subgroup, then $hK = Kh$ for any $h \in H$, and thus $HK = KH$ is a subgroup of $G$.
\end{lemma}
\begin{proof}
	We need to verify that $\forall h_1k_1 \cdot (h_2k_2)^{-1} = h_1k_1k_2^{-1}h_2^{-1}\in HK$. Since $h_1(k_1k_2^{-1}) \in HK = KH$, there exists $h,k$ such that $h_1k_1k_2^{-1} = kh$. 
	Then $khh_2^{-1} \in KH = HK$.
\end{proof}

\begin{remark}
	The converse of the above lemma is true, i.e. $HK$ is a subgroup $\implies$ $HK = KH$.
\end{remark}

\begin{lemma}
	If $H, K$ are both normal subgroups of $G$, then $HK$ is also a normal subgroup of $G$.
\end{lemma}
\begin{proof}
	$\forall g \in G$, we have $gHK = HgK = HKg$. 	
\end{proof}

\subsection{homomorphism}
\begin{definition}
	[homomorphism, isomorphism] $f: G \to G'$, for any $x,y \in G$, $f(xy) = f(x)f(y)$, then $f$ is called a homomorphism. If $f$ is bijection, then $f$ is an isomorphism.
\end{definition}
\begin{remark}
	$f: G\to G$ is injective (or surjective) and homomorphism, $f$ may not be a bijection (unless $G$ is finite.)	
\end{remark}

\begin{definition}
	[Kernel as a group homomorphism] For a homomorphism $\phi: G \to H$ of groups, the {\bf kernel} is 
	\[ker \;\phi = \{g \in G | \phi(g) = e_H\}\]
\end{definition}

\begin{lemma}
	Let $\phi:G\to H$ be a group homomorphism. \\
	(1) The image $\phi(G)$ is a subgroup of $H$. \\
	(2) The kernel $ker \; \phi$ is a normal subgroup of $G$.
\end{lemma}
\begin{proof}
	(1) It follows from that $\phi(g_1)\phi(g_2)^{-1} = \phi(g_1g_2^{-1}) \in \phi(G)$ \\
	(2) If $g_1, g_2 \in ker \; \phi$, then 
	\[\phi(g_1g_2^{-1}) = e_He_H^{-1} = e_H\]
	For any $g' \in G$, and any $g \in ker \ \phi$, \[\phi(g'gg'^{-1}) = \phi(g')e_H\phi(g')^{-1} = e_H\]  
\end{proof}

\begin{lemma}
	A homomorphism $\phi: G \to H$ of groups is injective if and only if $ker \; \phi = \{e_G\}$.
\end{lemma}

\begin{definition}
	[ring structure on endomorphisms of an abelian group] Let $M$ be an abelian group. Use $E(M)$ to denote endomorphisms of $M$ (naturally an abelian group). We upgrade $E(M)$ to be a ring (may not commutative) by defining 
	\[
		1 = id_M \\
		for \ f,g \in E(M), f\cdot g := f\circ g \in E(M)\]
\end{definition}
\begin{proposition}
	The above definition makes $E(M)$ into a ring.	
\end{proposition}
\begin{proof}
	The $(\cdot)$ operation on $E(M)$ forms a monoid. The distributivity law can be reduced to element-wise operation.
\end{proof}

\section{isomorphism theorems, composition series, statement of Holder Theorem}
\subsection{isomorphism theorems}

\begin{theorem}
	[The first isomorphism theorem] If $\phi: G \to H$ is a homomorphism of groups, then $ker \; \phi \unlhd G$ and \[
		G / ker \phi \cong \phi(G) \]
	In general, if $\phi: G\to H$ is a homomorphism, $\ker \phi \subseteq N \trianglelefteq G$, then $\phi(N) \unlhd \phi(G)$, and 
	\[
		G/N \cong \phi(G)/\phi(N)\]
\end{theorem}
\begin{proof}
	For any $\phi(g) \in \phi(G)$, $\phi(g)\phi(N)(\phi(G))^{-1} = \phi(gNg^{-1}) = \phi(N)$. Hence, $\phi(N)$ is normal.
	
	Define $f: G/N \to \phi(G) / \phi(N)$ by $f(gN) = \phi(g)\phi(N)$. 

	\begin{itemize}
		\item $f$ is well-defined. If $g = g'n, n \in N$, then $\phi(g)\phi(N) = \phi(g')\phi(n)\phi(N) = \phi(g')\phi(N)$.
		\item Can verify $f$ is homomorphism, injective. It's obvious $f$ is surjective.
	\end{itemize}
	Thus, $f$ is isomorphism.
\end{proof}

\begin{theorem}
	[The second homomorphism theorem] Let $G$ be a group, and let $A \leq G$ be a subgroup and $B \unlhd G$ a normal subgroup. Then $AB$ is a subgroup of G, $B \unlhd AB, A\cap B \unlhd A$, and 
	\[AB/B \cong A/(A\cap B)\]
\end{theorem}
\begin{proof}
	By lemma 1.2.2 we know $AB$ is a subgroup of $G$. \\
	For any $ab \in AB$, since $B$ is normal to $G$, $abB = aB = Ba$ and $aB = aBb = Bab$. So $B \unlhd AB$. \\
	It is clear that $A\cap B \leq A$. For any $a \in A, x \in A\cap B$, we have $axa^{-1} \in B$, since $B$ is normal. Also $axa^{-1} \in A$, since $x \in A$. So $A \cap B \unlhd A$. \\
	To show the isomorphism, we define $\phi: AB \to A/(A\cap B)$ by $\phi(ab) = a(A\cap B)$. It's easy to verify that $\phi$ is well-defined, surjective and a homomorphism, with $ker \phi = B$. By Theorem 1.3.1, we know the statement is true.
	\[\begin{tikzcd}
		AB && {A/(A\cap B)} \\
		\\
		& {AB/B}
		\arrow["\phi", two heads, from=1-1, to=1-3]
		\arrow["q"', from=1-1, to=3-2]
		\arrow["f"', from=3-2, to=1-3]
	\end{tikzcd}\]
\end{proof}

\begin{theorem}
	[The third isomorphism theorem] Let $G$ be a group and $H, K$ be normal subgroups with $H \leq K$. Then $K/H \unlhd G/H$, and 
	\[(G/H)/(K/H) \cong G/K\] 
\end{theorem}
\begin{proof}
	Consider the map 
	\[\begin{tikzcd}
		{\phi:} & {G/H} && {G/K} \\
		& gH && gK
		\arrow[from=1-2, to=1-4]
		\arrow[maps to, from=2-2, to=2-4]
	\end{tikzcd}\]
	\begin{itemize}
		\item $\phi$ is well-defined. We can simply redefine $\phi$ as $\phi(gH) = gH\cdot K = gK$ as product of subsets of $G$.
		\item $\phi$ is homomorphism. Easy to verify.
		\item $\phi$ is surjective.
		\item ker $\phi = \{gH | gK = K\} = \{gH | g \in K\} = K/H$. So $K/H \unlhd G/H$. And by the first isomorphism theorem, we statement holds.  
	\end{itemize}
\end{proof}

\begin{theorem}
	[The fourth isomorphism theorem/ Lattice isomorphism theorem] Let $G$ be a group and $N \unlhd G$. Then there is a bijection
	\[\begin{tikzcd}
		{\{subgroups \ of \ G \ containing \ N\}} && {\{subgroups \ of \ G/N\}} \\
		A && {A/N} \\
		{\pi^{-1}(\overline A)} && {\overline{A}}
		\arrow[tail reversed, from=1-1, to=1-3]
		\arrow[maps to, from=2-1, to=2-3]
		\arrow[maps to, from=3-3, to=3-1]
	\end{tikzcd}\]
	where $\pi: G \to G/N$ is the natural projection. \\
	This bijection preserves
	\begin{itemize}
		\item inclusion of groups 
		\item intersections
		\item normality of subgroups 
		\item quotients of subgroups 
	\end{itemize}
	Visually, we have: Lattice of subgroups of G containing N $\iff$ Lattice of subgroups of G/N. 
\end{theorem}

\section{Lattice}
\begin{definition}
	Let (S, $\leq$) be a set equipped with a partial order. (S, $\leq$) is called a {\it Lattice} if any $x, y \in S, \{x, y\}$ has a maximal lower bound and a minimal upper bound.  
	The lower bound is denoted by $x \wedge y$, while the upper bound is denoted by $x \vee y$.
\end{definition}

\begin{example}
	设 n 为正整数,$A_n$ 为 n 的所有正因数的集合,则 $A_n$ 关于整除关系构成格。
\end{example}
\begin{example}
	设 $P(B)$ 为 $B$ 的幂集,则 $P(B)$ 关于包含关系 $\subseteq$ 构成格,称为幂集格.
\end{example}

\begin{example}
	[子群格] 群G的所有子群,关于包含关系。
\end{example}

\section{composition series, Jordan-Holder Theorem, simplicity of An, direct product groups}
\begin{definition}
	[composition series] In a group $G$, a series of subgroups 
	\[\{0\} = N_0 \leq N_1 \leq ... \leq N_k = G\]	
	such that $N_{i-1} \unlhd N_i$ and $N_i / N_{i-1}$ is a simple group for $1 \leq i \leq k$ is called {\bf composition series}. In this case, $N_i / N_{i-1}$ is called a {\bf composition factor}.
\end{definition}

\begin{definition}
	[solvable] A group G is called {\bf solvable} if there exists a composition series 
	\[\{0\} = N_0 \leq N_1 \leq ... \leq N_k = G\] 
	such that $N_i / N_{i-1}$ is abelian.
\end{definition}

\begin{corollary}
	a finite group is solvable if and only if all the composition factors are $\mathbf{Z}_p$.
\end{corollary}

\begin{theorem}
	[Jordan-Holder] Let G be a non-trivial group, \\ 
	(1) G has a composition series. \\(2) Assume that a group $G$ has the following two composition series, 
	\[\{0\} = A_0 \leq A_1 \leq ... \leq A_m = G, \quad \{0\} = B_0 \leq B_1 \leq ... \leq B_n = G\]
	then $m = n$ and there exists a bijection $\sigma:\{1,2,...,m\}\to \{1,2,...,n\}$
	\[A_{\sigma(i)} / A_{\sigma(i-1)} \cong B_i / B_{i-1}\]
	for $i = 1, 2, ... m$
\end{theorem}

\begin{proof}
	{\it to be written}
\end{proof}

\subsection{The simplicity of An, n >= 5}

\begin{proposition}
	
\end{proposition}

\section{recognizing direct product, group actions, semi-direct product}
\subsection{recognizing direct products}
\begin{theorem}
	[criterion of direct product group] Suppose $G$ is a group with subgroups $H, K$ such that \\
	(1) $H, K$ are normal. \\
	(2) $H \cap K = \{1\}$ \\
	Then $HK \cong H \times K$
\end{theorem}
\begin{proof}
	Recall that Lemma 1.2.1 and 1.2.2 ensures that $HK = KH$ are normal subgroup of $G$. \\
	Consider the map 
	\[\begin{tikzcd}
		{\phi:} & {H\times K} && HK \\
		& {(h,k)} && hk
		\arrow[from=1-2, to=1-4]
		\arrow[maps to, from=2-2, to=2-4]
	\end{tikzcd}\]
	\begin{itemize}
		\item $\phi$ is a homomorphism. $\phi((h_1,k_1)(h_2,k_2)) = \phi((h_1h_2,k_1k_2))=h_1h_2k_1k_2$. It suffices to show that $h_2k_1 = k_1h_2$, or $h_2k_1h_2^{-1}k_1^{-1} = 1$. Since $h_2k_1h_2^{-1} \in K, k1h_2^{-1}k_1^{-1}\in H$, we know $h_2k_1h_2^{-1}k_1^{-1}\in H \cap K = \{1\}$.
		\item $\phi$ is surjective.
		\item $\ker \phi = \{(h,k) : hk = 1\} = \{(1,1)\}$.
	\end{itemize}
\end{proof}

\subsection{group actions}
\begin{definition}
	Let $G$ be a group and $X$ a set. A left G-action on $X$ is a map 
	
		\begin{align*}
			G \times X & \to X \\
			  (g,x) & \mapsto g  \cdot x 
	   \end{align*}
	satisfying the following conditions: \\
	(1) for any $x \in X$, $e \cdot x = x$ \\
	(2) for any $g,h \in G$ and $x \in X$, we have 
	\[g(hx) = (gh)x\]
\end{definition}

\begin{remark}
	for any $g\in G$, the induced $X \to X$ given by $x \mapsto g\cdot x$ is a bijection. Because the inverse is given by $x \mapsto g^{-1} x$.
\end{remark}

\begin{definition}
	[conjugate action] for $g\in G$, consider 
	\begin{align*}
		Ad_g: G \to G \\
		Ad_g(x) := gxg^{-1}
	\end{align*}
\end{definition}

\begin{proposition}
	Let $G$ be a group acting on a set $X$. Then we have a natural homomorphism from $G$ to the permutation group of $X$:
	\[\begin{tikzcd}
		{\Phi:} & G && {S_X} \\
		& g && {(\phi_g:x\mapsto g\cdot x)}
		\arrow[from=1-2, to=1-4]
		\arrow[maps to, from=2-2, to=2-4]
	\end{tikzcd}\]
	In fact, to given a group action is equivalent to give a homomorphism $\Phi:G\to S_X$.
\end{proposition}
\begin{proof}
	$\Phi(gh) = \phi_{gh}$, and $\phi_g \circ \phi_h (x) = g\cdot(h\cdot x) = (gh)\cdot x = \phi_{gh}(x)$.
\end{proof}

\begin{definition}
	(1) If the above $\Phi$ is injective, we say this action is {\bf faithful}. \\
	(2) If $\Phi$ is trivial, i.e. $\phi_g = id$ for any $g \in G$, we say the action is {\bf trivial}.
\end{definition}

\begin{theorem}
	[Cayley] Every group is isomorphic to a subgroup of some symmetry group. If $|G| = n$, then $G$ is isomorphic to a subgroup of $S_n$.
\end{theorem}
\begin{proof}
	Consider Prop. 1.6.2, it induces a homomorphism (injective) $G \to S_G$.
\end{proof}

\subsection{Automorphism groups}
\begin{definition}
	An {\bf automorphism} of a group $G$ is an isomorphism $\sigma: G \to G$. Then 
	\[Aut(G) := \{automorphisms \ of \ G\}\] 
	forms a group. It's a subgroup of $S_G$.
\end{definition}

\begin{definition}
	[Inner automorphism] 
	\[Inn(G) := \{Ad_g: g\in G\}\]
\end{definition}
\begin{proposition}
	\[Inn(G) \unlhd Aut(G)\]
\end{proposition}
\begin{proof}
	note that $\sigma Ad_g \sigma^{-1} = Ad_{\sigma(g)} $.
\end{proof}

\subsection{semi-direct products}

\section{Stabilizers, orbits of group actions, class equations}
\subsection{Stabilizers and orbits of group actions}

\begin{definition}
	Let $G$ be a group acting on a set $X$. For each $x \in X$,
	\begin{itemize}
		\item define the {\bf stabilizer subgroup} at $x$ to be $Stab_G(x)=\{g\in G| g\cdot x = x\}$
		\item define the {\bf orbit} of $x$ to be $Orb_G(x) = \{g\cdot x | g \in G\}\subseteq X$ 	
		\item define the {\bf fixed points} of set $X$ to be $X^G = \{x \in X | \forall g, gx = x\}$. Then for any $x \in X^G, Stab_G(x) = G$.
	\end{itemize}
\end{definition}

\begin{proposition}
	Let $G$ be a group acting on a set $X$ and $x\in X$. \\
	(1) $Stab_G(x)$ is a subgroup. \\
	(2) For $x,y \in X$, either $Orb_G(x) = Orb_G(y)$ or $Orb_G(x) \cap Orb_G(y) = \empty$. $X$ is the disjoint union of orbits for the $G$-action. \\
	(3) If $y \in Orb_G(x)$, i.e. $y = g\cdot x$ for some $g\in G$, then $Stab_G(y) = g \ Stab_G(x) \  g^{-1}$. Namely, the stabilizers at different points of an orbit are conjugate to each other.
\end{proposition}
\begin{proof}
	all very trivial.	
\end{proof}

\begin{example}
	[conjugacy classes] Definition 1.6.2 gives a group action of $G$ on itself. \\
	(1) If $G$ is abelian, the conjugacy class of $a\in G$ is just $\{a\}$. \\
	(2) For $G = GL_n(\mathbb{C})$, every matrix can be conjugated into a Jordan block. \[
		\{conjugacy \ classes \ of \ G\} \iff \{Jordan \ canonical\ form \ (with \ nonzero \ eigenvalues \ up \ to \ permutation)\} \] \\
	(3) $G = S_n$, the conjugacy classes are in one-to-one correspondence with partitions of $n = n_1 + n_2 + ... + n_t$.
\end{example}

\begin{definition}
	[centralizer, center, normalizer] Let $G$ be a group, $H$ a subgroup, and $S \subseteq G$ a subset.  \\
	(1) The subgroup $C_G(S) :=\{g\in G | for \ every \ s \in S, gsg^{-1} = s\}$ is called the {\bf centralizer} of $S$ in $G$\\
	(2) The subgroup $Z(G) :=\{g\in G | \forall h \in G, ghg^{-1} = h\} = C_G(G)$ is called the {\bf center} of $G$. \\
	(3) The subgroup $N_G(H) := \{g \in G | gHg^{-1} = H\}$ is called the {\bf normalizer} of $H$ in $G$.
\end{definition}

\begin{remark}
	Note that $Z(G)$ is abelian.
\end{remark}
\begin{proposition}
	(1) The Conjugation action induces a homomorphism $Ad: G \to Aut(G)$. Then $Z(G) = \ker (Ad)$. Thus, $Z(G)$ is a normal subgroup of $G$.
\end{proposition}
\begin{proof}
	$\ker (Ad) = \{g | Ad_g = id\}$	
\end{proof}

\begin{proposition}
	$G/Z(G) \cong Inn(G)$	
\end{proposition}
\begin{proof}
	Could be directly implied by Proposition 1.7.2 and fundamental homomorphism theorem.
\end{proof}

\begin{definition}
	[G-equivariant] Let $G$ be a group acting on two sets $X$ and $Y$. We say a map $\phi: X \to Y$ is G-equivalent if \\
	for all $g\in G, x \in X$, we have $\phi(g\cdot x) = g\cdot \phi(x)$.
\end{definition}

\begin{definition}
	[transitive] Let $G$ be a group acting on a set $X$. We say that the action is {\bf transitive} if \\
	for any $x,y \in X$, there exists $g \in G$, such that $x = gy$.
\end{definition}

\begin{proposition}
	If a group $G$ acts transitively on a set $X$, for every element $x \in X$, put $H = Stab_G(x)$. Then there is a G-equivalent bijection
	\[\begin{tikzcd}
		{\phi:} & {G/H} && X \\
		& gH && gx
		\arrow["\cong", from=1-2, to=1-4]
		\arrow[from=2-2, to=2-4]
	\end{tikzcd}\]
	(Here $G/H$ is not a quotient group, but simply equivalence class)
\end{proposition}
\begin{proof}
	verify that $\phi$ is well-defined, bijective, surjective, and preserves group action.
\end{proof}

\begin{corollary}
	Let $G$ be a group acting on a set $X$. For each $x \in X$, $G$ acts transitively on $Orb_G(x)$, thus we have
	\[Orb_G(x) \cong G/Stab_G(x)\] as $G-equivalence$.
	And 
	\[X\cong \bigsqcup_{G-orbits \ G\cdot x} G/Stab_G(x)\] 
\end{corollary}

\begin{corollary}
	If $X$ is a finite set, $G$ is a $p$-group acting on $X$, then
	\[|X| \equiv |X^G| \mod p\]
\end{corollary}
\begin{proof}
	A direct corollary from Corollary 1.7.4
\end{proof}

\subsection{class equations}
分类方程
\begin{theorem}
	[class equation] Let $G$ be a finite group (acting on itself by conjugation) \\
	(1) For each $g \in G$, the number of elements in its conjugacy class is \[|Ad_G(g)| = \frac {|G|} {C_G(g)} = [G:C_G(g)]\] \\
	(2) If $g_1, g_2,...,g_r$ are representatives of conjugacy classes of $G$ that are not contained in $Z(G)$, then 
	\[|G| = |Z(G)| + \sum_{i=1}^r [G : C_G(g_i)]\].
\end{theorem}

\begin{proposition}
	For a non-trivial $p$-group, $Z(G)$ is nontrivial. 
\end{proposition}
% \begin{proof}
% 	By (2) in Theorem 1.7.6, $p$ divides $|Z(G)|$.
% \end{proof}

\section{Sylow's Theorem}
\begin{definition}
	For $p$:prime, \\
	(1) A p-group is a finite group whose order is a power of $p$. \\
	(2) If $G$ is a finite group of order $|G|=p^rm$, and $p\nmid m$, a subgroup $H$ of $G$ of order exactly $p_r$ is called a {\bf Sylow p-subgroup}. write 
	\[Syl_p(G):=\{Sylow \ p-subgroup \ of \ G\} \quad and \quad n_p := |Syl_p(G)|\]
\end{definition}

\begin{theorem}
	[Sylow's theorem]
	Let $G$ be a finite group with $|G| = p_rm,p\nmid m$. \\
	\begin{itemize}
		\item (First Sylow Theorem) Sylow p-subgroup exists. 
		\item (Second Sylow Theorem) If $P$ is a Sylow p-subgroup, and $Q \leq G$ is of p-power order, then there exists $g \in G$ such that $Q \leq gPg^{-1}$ (note that $gPg^{-1}$ is also a Sylow p-subgroup).
		\item In other words, we have \begin{itemize}
			\item all Sylow p-subgroups are conjugate.
			\item all subgroups of p-power order is contained in a Sylow p-subgroup.
		\end{itemize}
		\item (Third Sylow Theorem) $n_p = |Syl_p(G)|$ satisfies \\
		(1) $n_p \equiv 1\mod p$ \\
		(2) $n_p \mid m$
	\end{itemize}	
\end{theorem}

\begin{proof}
	[proof of first Sylow Theorem -- version 1]
	Induce on $|G|$. When $|G| = 1$, trivial. \\
	Suppose that the theorem is proved for finite grouops of order $< n$. Let $G$ be a finite group of order $n=p^rm, p \nmid m$. \\
	\underline{Case 1}: If $r = 0$, trivial.
	\underline{Case 2}: If $p | |Z(G)|$, then $Z(G)$ is a finitely generated abelian group. So 
	\[Z(G) = \mathbb{Z}_p^{r_1} \times ... \times \mathbb{Z}_p^{r_s} \times ...\]
	We write $Z(G)_p$ for the p-part of $Z(G)$; then $|Z(G)_p| = p^{r'}$ for some $r' \geq 1$. \\
	Consider the quotient homomorphism \[\begin{tikzcd}
		G && {G/Z(G)_p=:\overline{G}}
		\arrow["\pi", two heads, from=1-1, to=1-3]
	\end{tikzcd}\]
	where the quotient $\overline{G}$ has order $p^{r-r'}m<n$. By inductive hypothesis, $\overline{G}$ contains a Sylow p-subgroup $H := \overline{H}$ of order $p^{r-r'}$. Then $\pi^{-1}(\overline{H})$ is a subgroup of $G$ (By the fourth isomorphism theorem) of order 
	\[|\overline{H}|\cdot |\ker \pi| = p^r.\]
	So $H$ is a Sylow p-subgroup of $G$.  \\
	\underline{Case 3}: If $p \nmid Z(G)$ but $p \mid G$. \\
	Then class equation 
	\[|G| = |Z(G)| + \sum_{i=1}^t [G:C_G(g_i)]\]
	follows that there exists some $i$ such that $[G : C_G(g_i)]$ is not divisible by $p$. Thus $|C_G(p_i)|$ has order $p^r m'$ for some $m' \mid m$ but $m' \neq m$.
	By inductive hypothesis we know there exists a Sylow p-subgroup $H$ of $C_G(g_i)$, which is also a subgroup of $G$.
\end{proof} 

\begin{remark}
	The above proof can be easily modified to show a stronger result:\\
	If $|G| = p^rm$, then for every $0 \leq k \leq r$, there exists some subgroup $H \leq G$, such that $|H| = p^k$.
\end{remark}

Now introduce a similar theorem, which is also an application of group action and orbit decomposition.
\begin{theorem}
	[A. L. Cauchy] $G$ is finite, $p$ is a divisor of $|G|$, then there exists $g \in G$ such that $ord(g) = p$.
\end{theorem}
\begin{proof}
	Let $H = \mathbb{Z}/p\mathbb{Z}$ (note that it's a p-group) acts on the following set
	\[X = \{(g_1,...,g_p) | g_1...g_p = 1\}\]
	Since $g_p$ is uniquely determined by $(g_1,...,g_{p-1})$, we know $|X| = p^{p-1}$.
	We define the group action by 
	\[\overline{k} \cdot (g_1,...,g_p) = (g_{1+k},g_{2+k},...,g_{p+k})\]
	The "$+$" operation in the index is under modulo.
	The fixed points of $X$ is 
	\[X^H = \{(g_1,...,g_p) \in X | \forall \overline{k}, \overline{k}\cdot (g_1,...,g_p) = (g_1,...,g_p)\}=\{(g,g,...,g)\in X\} = \{(g,...,g) | g^p = 1\} \]
	Since $(1,...,1) \in X^H$, $X^H$ is not empty.

	By Corollary 1.7.5, we know $|X| \equiv |X^H| (mod p)$, which implies $|X^H| \equiv 0 (mod p)$. Hence, there exists some $g \in G$, $ord(g) = p$.
\end{proof}

We also use group action to prove the second Sylow Theorem.
\begin{proof}
	[proof of the second Sylow Theorem] Let $P \leq G$ be a Sylow p-subgroup, $Q \leq G$ a subgroup of p-power order. 

	When $|Q| = 1$, done.

	Now assume $|Q| = p^{r'}$ with $r' \geq 1$. Consider the translation action of $Q$ on $G/P$
	\[Q\curvearrowright G/P\]
	by $q\cdot gP := qgP$.

	Then we have 
	\[|G/P| = \sum_{i=1}^t |Q/Stab_i|\]
	Since the left side is not divisible by $p$, there exists some $i$ such that $[Q : Stab_i]$ is not divisible by $p$. Let 
	\[Q' = \{q\in Q | qgP = gP\} = Stab_i \leq Q\]
	Then $|Q'| = p^{r'} = |Q|$. So $Q' = Q$. For any $q \in Q$, we have $qgP = gP \implies qg \in gP \implies q \in gPg^{-1}$.
	So we deduce that $Q \leq gPg^{-1}$. 
\end{proof}

\begin{corollary}
	All Sylow subgroups are conjugate.
\end{corollary}
\begin{proof}
	Note that $gPg^{-1}$ is also a Sylow p-subgroup if $P \in Syl_p(G)$.
\end{proof}

\begin{corollary}
	$|Syl_p(G)| = 1 \iff $ there is a Sylow p-subgroup $P$ is normal.
\end{corollary}
\begin{proof}
	By Corollary 1.7.10, it's trivial.
\end{proof}

\begin{corollary}
	If $P$ is a Sylow p-subgroup, then $N_G(N_G(P)) = N_G(P)$, and $N_G(P)$ contains a unique Sylow p-subgroup, which is $P$.
\end{corollary}
\begin{proof}
	Since $P \unlhd N_G(P)$, by Corollary 1.7.11, $N_G(P)$ contains a unique normal Sylow p-subgroup. 

	$P$ is a group, so $N_G(P) \subseteq N_G(N_G(P))$. 
	
	For any $g\in G$ such that $gN_G(P)g^{-1} = N_G(P)$, we have $gPg^{-1} \in N_G(P)$ is a Sylow p-subgroup in $N_G(P)$. Thus, $gPg^{-1} = P$,
	which is equivalent to $g \in N_G(P)$.
\end{proof}

\begin{proof}
	[proof of the third Sylow Theorem] 
	(1) Consider the conjugation action of $G$ on $Syl_p(G)$. By second Sylow theorem we know this action is {\bf transitive} (There is only one orbit). From this, we deduce that for some $P \in Syl_p(G)$
	\[n_p = |Syl_p(G)| = \frac {|G|} {|N_G(P)|}= \frac {p^r \cdot m} {p^r \cdot [N_G(P) : P]}\] 
	Thus, $n_p | m$.

	(2) Choose any Sylow p-subgroup $P$. Consider the conjugation action of $P$ on $Syl_p(G)$. Then we have 
	\[n_p = \sum_{orbits \ Ad_P(P_i)} |P/Stab_P(P_i)|\]
	If $Stab_P(P_i) \neq P$, then $ p \mid |P/Stab_P(P_i)|$. 

	If $Stab_P(P_i) = P$, then $P \subseteq N_G(P_i)$. By Corollary 1.7.12 we know there is a unique Sylow p-subgroup in $N_G(P_i)$. So $P = P_i$.
	It follows that $n_p \equiv 1 (mod \ p)$.
\end{proof}

\subsection{Applications of Sylow's theorem}

\section{Exercises}

\begin{proposition}
	Define $End(G) := Hom(G, G)$. Prove that $End(\mathbb{Z}_n) \cong \mathbb{Z}_n$. Consider $f_p : x \mapsto px$ under modulo operation.
\end{proposition}

\begin{proposition}
	$N \le G$, $[G : N] = 2$, then $N \unlhd G$.
\end{proposition}

\begin{proposition}
	$|G| = pm$, where $p$ is prime and $m < p$. Prove that any $p$-ordered subgroup of $G$ is normal. In fact, by the third Sylow's Theorem, $G$ has a unique Sylow-$p$ subgroup, which is normal.
\end{proposition}

\begin{proof}
	[hint] Let $H \le G, |H|=p$, we assert $xHx^{-1} = H$. If not, use Lagrange's Theorem to show that $H \cap xHx^{-1} = e$. By Theorem 1.6.1, we know $|HK| = p^2 > |G|$.
\end{proof}

%%% -------------------------
%%%  chapter{Rings and Ideals}
%%%
\chapterimage{skybox_resized.png} % Chapter heading image
\chapter{Rings and Fields}
\begin{definition}
	[ring] $(R,+,\cdot)$ is called a ring if 
	\begin{enumerate}
		\item $(R, +)$ is an abelian group.
		\item $(R, \cdot)$ is a semigroup.
		\item left and right distributivity law of $\cdot$ over $+$.
	\end{enumerate}
\end{definition}

\begin{definition}
	[commutative ring, 幺环, zero-divisor, integral domain, division ring]
\end{definition}

\begin{definition}
	[field] A commutative ring $R$ with an identity element such that $R^{\times}$ is a group under $\cdot$.
\end{definition}

\begin{theorem}
	A non-trivial finite ring without zero divisor is a division ring.
\end{theorem}

\begin{corollary}
	A finite integral domain is a field.	
\end{corollary}

\begin{definition}
	[characteristic of a ring without zero divisor]
\end{definition}

\begin{definition}
	[left(right) ideal] Let $R$ be a ring, $I \subseteq R$ and $(I, +)$ is a subgroup of $(R, +)$. 
	If $RI \subseteq I$, then $I$ is called a left ideal; If $IR \subseteq I$, then $I$ is called a right ideal.
	If $I$ is both a left and right ideal, then $I$ is called an ideal.
\end{definition}

\begin{definition}
	[generated ideal] Let $R$ be a ring, $\emptyset \neq T \subseteq R$, define $\langle T\rangle = \cap \{I : T \subseteq I, I \text{ is an ideal of } R\} $.
	When $T = \{a\}$, we use $\langle a \rangle$ to denote the {\bf principal ideal} generated by $a$.
\end{definition}

\begin{theorem}
	$R$ is a ring, then 
	\[
		\langle a \rangle = \{\sum_i x_i a y_i + sa + at + n\cdot a : \forall x_i,y_i,s,t \in R, n\in \mathbb{Z}\}\]
\end{theorem}

\begin{corollary}
	Let $R$ be a ring,
	\begin{itemize}
		\item when $R$ is commutative, $\langle a \rangle = \{sa + na : s\in R, n \in \mathbb{Z}\}$.
		\item when $R$ contains $1$, $\langle a \rangle = \{\sum_i x_iay_i\}$.
		\item when $R$ contains $1$ and is commutative, $\langle a \rangle = Ra$.
	\end{itemize}
\end{corollary}
%%% -------------------------
%%%  chapter{Rings and Ideals}
%%%
\chapterimage{destiny2_resized.jpg} % Chapter heading image
\chapter{Communitative Rings and Ideals}
If not pointed out specifically, the notion "ring" refers to a commutative ring with an identity element. 

\section{rings, ideals, quotient rings}
\begin{definition}
	[ring homomorphism] Let $A, B$ be rings, $f : A\to B$ is a homomorphism when \\
	(1) $f(x+y) = f(x) + f(y)$. So $f$ is a homomorphism of abelian groups. \\
	(2) $f(xy) = f(x)f(y)$, $f(1) = 1$. So $f$ is a homomorphism between the monoids $(A, \cdot)$ and $(B, \cdot)$.

\end{definition}
\begin{definition}
	[ideal of a ring] An ideal $I$ of a ring $A$ is an additive subgroup and is such that $A\I \subseteq I$.
\end{definition}
\begin{example}
	Every ring $A$ has 2 trivial ideals: $\{0\}$ and $A$.
\end{example}

Below, $I$ denotes the ideal of ring $A$.
\begin{definition}
	[quotient ring] Define multiplication in the quotient group $A/\I$ by\\
	\[(a + I) \cdot (b + I) = ab + I\]
	It is well defined. Now $A/I$ is made into a ring called the {\it quotient ring}. The mapping $\phi: A \to A/I$ which maps each $x \in A$ to its coset $x + I$ is a surjective ring homomorphism.
\end{definition}

\begin{proposition}
	There is a one-to-one order preserving correspondence between 
	\[\begin{tikzcd}
		{\{J|I\subseteq J \subseteq A, J:ideal\}} && {\{\overline{J} | ideal \; \overline{J} \subseteq A/I\}} \\
		J && {J + I} \\
		{\phi^{-1}(\overline{J})} && {\overline{J}}
		\arrow["{1:1}", tail reversed, from=1-1, to=1-3]
		\arrow[maps to, from=2-1, to=2-3]
		\arrow[maps to, from=3-3, to=3-1]
	\end{tikzcd}\]
\end{proposition}
\begin{proof}
	First, Let's show that $J + I$ is an ideal in $A/I$. \\
	$J + I$ is abelian : trivial; $\forall x + I \in A/I, (x+I)\cdot(J+I) = (Jx + I) \subseteq (J+I) $, since $J$ is an ideal. \\
	Second, we can verify this mapping to be invertible.

\end{proof}

\begin{corollary}
	If $f: A\to B$ is any ring homomorphism, the {\it kernel} of $f(=f^{-1}(0))$ is an ideal of $A$, and the image of $f (= f(A))$ is a subring $C$ of $B$, but may not be an ideal.
\end{corollary}
\begin{proof}
	Consider the embedding mapping\[\begin{tikzcd}
		{\mathbb{Q} } && {\mathbb{Q}[X]}
		\arrow[hook, from=1-1, to=1-3]
	\end{tikzcd}\]
	The image is absolutely not an ideal.
\end{proof}

\begin{theorem}
	[fundamental homomorphism theorem] $f: A \to B$ is a ring homomorphism, $I$ is the kernel of $f$, $g(a + I) := f(a)$ then $g$ is a ring isomorphism.
	\[\begin{tikzcd}
		A && {Im(f)} && B \\
		\\
		&& {A/I}
		\arrow["f", two heads, from=1-1, to=1-3]
		\arrow[hook, from=1-3, to=1-5]
		\arrow["g", from=3-3, to=1-3]
		\arrow["\phi", from=1-1, to=3-3]
	\end{tikzcd}\]

\end{theorem}

\section{Chinese Remainder Theorem}
\begin{theorem}
	Let $N \in \mathbb{N}^+, N = n_1n_2...n_k$, where $n_i,n_j (i \neq j)$ are coprime. We have 
	\begin{align*}
	\phi:  \mathbb{Z}/N\mathbb{Z} &\to \prod_{i = 1}^{k}\mathbb{Z}/n_i\mathbb{Z}\\
	 [x]_N &\mapsto ([x_i]_{n_i})_{i=1}^k
	\end{align*}
	is an isomorphism of rings.
\end{theorem}

\section{zero-divisors, nilpotent elements, units}
\begin{definition}
	[zero-divisor] a zero-divisor in a ring $A$ is an element $x$ for which there exists $y \neq 0$ in $A$ such that $xy = 0$
\end{definition}
\begin{definition}
	[integral domain] a ring with no zero-divisors $\neq 0$ and not a zero ring.
\end{definition}

\begin{definition}
	[nilpotent] An element $x \in A$ is {\it nilpotent} if $x^n = 0$ for some $n > 0$.
\end{definition}
\begin{remark}
	A nilpotent element is a zero-divisor.
\end{remark}

\begin{definition}
	[unit 可逆元] A unit in $A$ is an element $x$ such that $xy = 1$ for some $y \in A$. Note that $y$ is uniquely determined by $x$, and is written as $x^{-1}$. 
\end{definition}
\begin{remark}
	The units in $A$ form a abelian group under multiplication.
\end{remark}

\begin{definition}
	[field] A field is a ring $A$ which $1 \neq 0$ and every non-zero elem. is a unit.
\end{definition}

\begin{proposition}
Let A be a ring $\neq 0$. The following are equivalent:\\
(1) A is a field;\\
(2) The only ideals in A are ${0}$ and $(1)$; \\
(3) Every non-trivial homomorphism of A into a non-zero ring B is injective.	
\end{proposition}

\section{prime ideals and maximal ideals}
\newcommand{\fp}{\mathfrak{p}}
\newcommand{\fm}{$\mathfrak{m}$}
\begin{definition}
	[prime ideal]An ideal $\fp$ in $A$ is {\it prime} if $\fp \neq (1)$ and if $xy \in \fp \implies x\in \fp$ or $y \in \mathfrak{p}$ 
\end{definition}

\begin{definition}
	[maximal ideal] An ideal \fm in $A$ is {\it maximal} if \fm $\neq (1)$ and if there is no ideal $\alpha $ such that \fm $\subset \alpha \subset (1)$(strict inclusion).
\end{definition}
\begin{remark}
	\fm \text{ can} be $\{0\}$.
\end{remark}

\begin{proposition}
	$\fp$ is prime $\iff$ $A/\fp$ is an integral domain.
\end{proposition}
\begin{proof}
	Easy to verify.
\end{proof}

\begin{proposition}
	\fm \text{ }is maximal $\iff$ $A/m$ is a field. Hence, a maximal ideal is prime.
\end{proposition}
\begin{proof}
	By Proposition 2.1.1 and Proposition 2.2.1, the statement holds.
\end{proof}

\begin{proposition}
	If $f: A\to B$ is a ring homomorphism and $q$ is a prime ideal of $B$, then $f^{-1}(q)$ is a prime ideal in $A$.
\end{proposition}
\begin{proof}
	If $a, b \in A$ such that $f(a) = f(b) \in q$. Then $f(a-b) = f(a)-f(b) \in q$. Thus, $f^{-1}(q)$ is abelian. For any $a \in f^{-1}(q), x \in A$, we have 
	$f(ax) = f(a)f(x) \in Bq = q$. Thus, $f^{-1}(q)$ is an ideal. For any $a, b\in A, ab \in f^{-1}(q) \iff f(ab) \in q \iff f(a)\cdot f(b)\in q \iff f(a) \in q \vee f(b) \in q \iff a \in f^{-1}(q) \vee b \in f^{-1}(q) \iff f^{-1}(q)$ is a prime ideal.  	
\end{proof}

\begin{remark}
	If $m$ is a maximal ideal of $B$, it is not necessarily true that $f^{-1}(m)$ is maximal in $A$. Consider $A = \mathbb{Z}, B = \mathbb{Q}, m = \{0\}$.
\end{remark}

\begin{theorem}
	Every ring $A \neq 0$ has at least one maximal ideal.
\end{theorem}
This theorem relys on Zorn's Lemma. We first introduce it.

\begin{definition}
	[chain in a partially ordered set] Let $S$ be a non-empty partially ordered set. A subset $T$ of $S$ is a chain if either $x \leq y$ or $y \leq x$ for every pair of elements in $T$.
\end{definition}

\begin{lemma}
	[Zorn] If every chain $T$ of $S$ has an upper bound in $S$, then $S$ has at least one maximal element. Zorn's Lemma is equivalent to the axiom of choice.
\end{lemma}

\begin{proof}
	Let's prove theorem 2.3.4, using Zorn's Lemma. \\
	Let $\Sigma = \{I : I \ is \ ideal, I \neq (1)\}$. Order $\Sigma$ by inclusion. $\Sigma$ is not empty, since $0 \in \Sigma$. For each chain, consider the union as another ideal $\neq (1)$ to be an upper bound. Then Zorn's lemma yields that there is a maximal element.
\end{proof}
\begin{remark}
	If $A$ is Noetherian, we can avoid the use of Zorn's lemma.	
\end{remark}

\begin{corollary}
	If $a \neq (1)$ is an ideal of $A$, there exists a maximal ideal of $A$ containing $a$.
\end{corollary}
\begin{proof}
	Replace $\Sigma$ by $\{I: I \ is \ ideal \ containing \ a, I \neq (1)\}$ in the proof of Theorem 2.3.4 .
\end{proof}

\begin{corollary}
	Every non-unit of $A$ is contained in a maximal ideal.
\end{corollary}

\begin{definition}
	[local ring, residue field] If a ring $A$ has exactly one maximal ideal $m$ (e.g. fields), then $A$ is called a {\it local ring}. The field $k = A/m$ is called the residue field of $A$.
\end{definition}

\begin{proposition}
	Let $A$ be a ring and $m \neq (1)$ an ideal of $A$ such that $\forall x \in A - m$ is a unit in $A$. Then $A$ is a local ring and $m$ its maximal ideal.
\end{proposition}

First, we observe the following
\begin{lemma}
	Every element in a maximal ideal is not a unit.
\end{lemma}

\begin{proof}[proof of Proposition 2.3.8]
	From corollary 2.3.6 and lemma 2.3.9 we know $m$ is a maximal ideal. Also from lemma 2.3.9, we know there doesn't exists other maximal ideals. Thus, $A$ is a local ring.
\end{proof}

\begin{proposition}
	Let $A$ be a ring and $m$ a maximal ideal, such that every element of $1 + m$ is a unit in $A$. Then $A$ is a local ring.
\end{proposition}
\begin{proof}
	Make an analogy to Bezout Theorem. Let $x \in A - m$. Since $m$ is maximal, the ideal generated by $x$ and $m$ is $(1)$, hence there exists $y \in A, t\in m$ such that $xy + t = 1$.
	Thus $xy = 1 - t \in 1 + m$, which means $x$ is a unit. 
\end{proof}

\begin{example}
	$A = F[X_1,...,X_n], F$ : field. Let $f \in A$ be an irreducible polynomial. By unique factorization, the ideal $(f)$ is prime. When $n \geq 2$, it's not a {\it principal ideal domain}.	
\end{example}

\begin{example}
	Every ideal in $\mathbf{Z}$ is of the form $(m)$ for some $m \geq 0$. The ideal is prime $\iff$ $m = 0$ or is a prime number. For all ideals $(p)$ are maximal.
\end{example}

\begin{definition}
	[principal integral domain] an integral domain where every ideal is principal.
\end{definition}

\begin{proposition}
	Every non-zero prime ideal is maximal in principal integral domain.
\end{proposition}
\begin{proof}
	[Hint] The cancellation law applies in the integral domain. \\
	Let $(x)$ be a prime ideal and $(x) \subset (y)$. Then $x = yz$ for some $z$. $since y \notin (x)$, we know $z \in (x)$. Thus $z = xt$ and $x = ytx$, which implies $yt = 1$, and $(y) = 1$.
\end{proof}

\section{nilradical and Jacobson radical}
\begin{proposition}
	The set $\mathfrak{R}$ of all nilpotent elements in a ring $A$ is an ideal, and $A / \mathfrak{R}$ has no nilpotent element $\neq 0$.
\end{proposition}
\begin{proof}
	For any $x,y \in \mathfrak{R}$, there exists $n \geq 0$ such that, $(x-y)^n = 0$. Thus, $x - y \in \mathfrak{R}$ and $\mathfrak{R}$ is abelian group. It's easy to show that $\mathfrak{R}$ is an ideal.
	If there exists $a \in A$, such that $\exists n > 0, (a + \mathfrak{R})^n = 0 = a^n + \mathfrak{R}$ , then $a \in \mathfrak{R}$. Hence, $A / \mathfrak{R}$ has no non-zero nilpotent element.
\end{proof}

The ideal $\mathfrak{R}$ is called the {\it nilradical} of $A$.
\begin{proposition}
	The nilradical of $A$ is the intersection of all the prime ideals of $A$.
\end{proposition}
\begin{proof}
	We observe that every nilpotent element belongs to any prime ideal. Hence, $\mathfrak{R} \subseteq \cap_{p:prime \ ideal}p$.
	On the other side, for each element within the intersection of all prime ideals, 试图用Zorn's lemma寻找一个极大理想,证明这也是一个prime ideal. 从而non-nilpotent element不属于这个ideal.
\end{proof}

\begin{definition}
	[Jacobson radical] The Jacobson radical $\mathfrak{R}$ of $A$ is defined to be the intersection of all the maximal ideals of $A$.
\end{definition}
It can be characterized as
\begin{proposition}
	$x\in \mathfrak{R} \iff 1 - xy$ is a unit for all $y \in A$.
\end{proposition}
\begin{proof}
	$\implies$: Suppose $1-xy$ is not a unit. By corollary 2.3.7 it belongs to some maximal ideal $m$. But $x\in \mathfrak{R}\subseteq m$, hence $xy \in m$ and $1 \in m$, which is absurd. \\
	$\Leftarrow$: 考虑Bezout定理。If $x\notin m$ for some maximal ideal $m$, then $m + (x)$ generate the unit ideal $(1)$, so that $u + xy = 1$ for some $u \in m, y \in A$. Hence $1 - xy \in m$ is not a unit.
\end{proof}

\section{operations on ideals}
\begin{definition}
	[intersection] the ideal $A \cap B$
\end{definition}
\begin{remark}
	The union of $A, B$ is typically not an ideal.
\end{remark}

\begin{definition}
	[sum] the ideal $A + B$
\end{definition}

\begin{definition}
	[product] $AB$ denotes the ideal generated by elements in set $AB$, i.e. $AB = \{\sum_{finite}a_ib_i : a_i \in A, b_i \in B\}$	
\end{definition}

\begin{definition}
	[coprime] ideals $A,B$ are coprime if $A + B = (1)$.
\end{definition}
\begin{remark}
	different prime ideals are not necessarily coprime. For example, let $A = F[X, Y], p_1 = (X), p_2 = (Y)$.	
\end{remark}

\begin{definition}
	Let $A$ be a ring and $\alpha_1, ... ,\alpha_n$ ideals of $A$. Define a homomorphism
\[\phi: A \to \prod_{i=1}^n(A/\alpha_i)\]
by the rule $\phi(x) = (x+\alpha_1,...,x+\alpha_n)$.

\end{definition}
\begin{remark}
	Let $a, b$ be ideals of ring $A$, then $ab \subseteq a \cap b$
\end{remark}

\begin{proposition}
	(1) If $a_i, a_j$ are coprime whenever $i\neq j$, then $\prod a_i = \cap a_i$. \\
	(2) $\phi$ is surjective $\iff$ $a_i, a_j$ are coprime whenever $i\neq j$. \\
	(3) $\phi$ is injective $\iff$ $\cap a_i = (0)$
\end{proposition}
\begin{proof}
	The third statement can be shown by $\ker \phi = \cap \alpha_i$
\end{proof}
\begin{remark}
	(2) is the generalized form of Chinese Remainder Theorem.	
\end{remark}

\begin{proposition}
	Let $p_1, ... , p_n$ be prime ideals and let $\alpha$ be an ideal contained in $\cup_{i=1}^n p_i$. Then $\alpha \subseteq p_i$ for some $i$. 
\end{proposition}
\begin{proof}
	Prove by induction on n in the form 
	\[a \nsubseteq p_i (1\leq i \leq n) \implies a\nsubseteq \cup_{i=1}^n p_i\]
	n = 1 : trivial. If $n>1$ and the result is true for n - 1, then for each $i$ there exists $x_i \in a$ such that $x_i \notin p_j(\forall j\neq i)$. If there is some $i$ such that $x_i \notin p_i$, succeed. If not, then $x_i \in p_i$ for all $i$,
	consider
	\[y = \sum_{i=1}^{n}x_1x_2...x_{i-1}x_{i+1}...x_n\].
\end{proof}


\begin{proposition}
	Let $a_1,...a_n$ be ideals and $p$ be a prime ideal, $p \supseteq \cap_{i=1}^n a_i$. Then $p \supseteq a_i$ for some $i$. If $p = \cap a_i$, then $p = a_i$ for some $i$.
\end{proposition}
\begin{proof}
	We observe that if $x,y \notin p$, then $xy \notin p$.
\end{proof}

\begin{definition}
	[ideal quotient]
	If $a, b$ are ideals in a ring $A$m their {\it ideal quotient} is 
	\[(a : b) = \{x\in A : xb\subseteq a\}\]
\end{definition}
\begin{remark}
	$(a :b)$ is an ideal.
\end{remark}

\begin{definition}
	[annihilator] $(0:b)$ is called the {\it annihilator} of $b$ and denoted by $Ann(b)$.
\end{definition}

\section{extension and contraction}
let $f: A \to B$ be a ring homomorphism.
\begin{definition}
	[extension]
	If $a$ is an ideal in $A$, we define the {\it extension} $a^e$ to be the ideal generated by $f(a)$ in $B$. Explicitly, $a^e$ is the set of all sums $\sum y_if(x_i)$ where $x_i \in a, y_i \in B$.
\end{definition}

\begin{definition}
	[contraction] If $b$ is an (prime) ideal of $B$, then $f^{-1}(b)$ is always an (prime) ideal of $A$, called the contraction $b^c$ of $b$.
\end{definition}
To show its correctness, we have the following
\begin{proposition}
	Let $f:A \to B$ be a surjective ring homomorphism. There is a one-to-one correspondence between the ideals of $f(A) = B$ and ideals of $A$ which contain $\ker f$, and prime ideals correspond to prime ideals.
	\[\begin{tikzcd}
		{\{ideals \ of \ A : A \supseteq \ker f\}} && {\{ideals \ of \ B\}} \\
		I && {f(I)} \\
		{f^{-1}(J)} && J
		\arrow["{1:1}", tail reversed, from=1-1, to=1-3]
		\arrow[maps to, from=2-1, to=2-3]
		\arrow[maps to, from=3-3, to=3-1]
	\end{tikzcd}\]
\end{proposition}
\begin{proof}
	We only show that prime-ideal correspondence. If $I$ is prime, for any $f(a), f(b)$ where $a,b\in I$, $f(a)f(b) \in f(I) \iff f(ab) \in f(I) \iff ab \in I \iff a\in I \ or \ b \in I \iff f(a) \in f(I) \ or \ f(b) \in f(I)$. Thus, $f(I)$ is prime. The other side is similar.
\end{proof}

\section{polynomial rings}
Here, we mainly consider integral domain or field to be the ring. We will use the notion of {\bf degree}.
\begin{lemma}
	Let $R$ be an integral domain. For all non-zero $f, g \in R[X]$ we have $\deg (fg) = \deg f + \deg g$. And $R[X]$ is also an integral domain, with $R[X]^{\times} = R^{\times}$.
\end{lemma}

Now about polynomials over a field $F$.
\begin{proposition}
	[带余除法] For any $a,d \in F[X], d \neq 0$, there exists unique $q,r \in F[X]$ such that $\deg (r) <\deg (d), a = dq + r$. Here, we define $deg (0) = -\infty$.
\end{proposition}
\begin{proof}
	To find $r$, consider set $\{a - dq : q \in F[X]\}$. There exists element such that $\deg(a - dq)$ is minimal.	
\end{proof}

\begin{definition}
	[root] For a commutative ring $R$, $f \in R[X], a \in R$ such that $f(a) = 0$. Then $a$ is called a root of $f$.
\end{definition}

By proposition 2.7.2, we immediately get 
\begin{proposition}
	$f(a) = 0 \iff (X - a) | f$.
\end{proposition}

As to the number of roots, we have 
\begin{proposition}
	$F$ : field, $f \in F[X]-\{0\}$, then $f$ has at most $\deg f$ roots in $F$.
\end{proposition}
\begin{proof}
	Use proposition 2.7.3 and induce on the degree of $f$.
\end{proof}

\begin{definition}
	[Fraction Field of an integral domain] Let $A$ be an integral domain, use {\it Frac(A)} to denote ... 
\end{definition}

With fraction field, we can extend proposition 2.7.4,
\begin{lemma}
	Let $R$ be an integral domain, $f \in R[X] - \{0\}$, then $f$ has at most $\deg f$ different roots in $R$.	
\end{lemma}

\chapterimage{town_resized.png} % Chapter heading image
\chapter{Module Theory}

\section{modules and module homomorphisms}
\begin{definition}
	Let $A$ be a communitative ring. An $A-$module is an abelian group $M$ on which $A$ acts linearly, i.e. for any $a,b \in A, x,y \in M$
	\[
		a(x+y) = ax + ay \\
		(a+b)x = ax + bx \\
		(ab)x = a(bx) \\
		1x = x\]
\end{definition}

\begin{proposition}
	There is a one-on-one correspondence between all $A$-module structure on $M$ (denoted as $Mod_A(M)$) and ring homomorphisms $Hom(A, E(M))$.
\end{proposition}
\begin{proof}
	check both sides.
	\[\begin{tikzcd}
		{Mod_A(M)} && {Hom(A,E(M))} \\
		M && {(a\mapsto(m\mapsto am))} \\
		{a\cdot m := f(a)(m)} && f
		\arrow["{1:1}", tail reversed, from=1-1, to=1-3]
		\arrow[maps to, from=2-1, to=2-3]
		\arrow[maps to, from=3-3, to=3-1]
	\end{tikzcd}\]
\end{proof}

\begin{example}
	1) An ideal $\alpha$ of $A$ is an $A$-module.

	2) If $A$ is a field, then $A-$module = vector space.

	3) If $A=k[X]$ where $k$ is a field, then $A-module$ is a $k-$ vector space with a linear transformation.

	4) If $A = \mathbb{Z}$ then $Z-module$ = Abelian group.
\end{example}

\begin{definition}
	[module homomorphism] Let $M, N$ be $A-$modules. A mapping $f:M \to N$ is an $A-$homomorphism if 
	\[
		f(x+y) = f(x) + f(y) \\
		f(ax) = a\cdot f(x)\]
\end{definition}

\begin{definition}
	The set of all $A$-module homomorphisms from $M$ to $N$ can be turned into an $A-$module as follows \[
		(f+g)(x) := f(x) + g(x) \\
		(af)(x) := a(f(x))\]
	This $A$-module is denoted by $Hom_A(M, N)$
\end{definition}

\section{submodules and quotient modules}
\begin{definition}
	A submodule $M'$ of $M$ is a subgroup of $M$ which is closed under $A$-action.
\end{definition}

\begin{definition}
	[quotient] The abelian group $M/M'$ inherits an $A-module$ structure by defining 
	\[
		a(x + M') = ax + M'\]
	The quotient map $\pi: M \to M/M'$ is an $A$-module homomorphism.
\end{definition}

\begin{proposition}
	[A generalization of lattice homomorphism theorem in ring ideals] There is a one-to-one order-preserving correspondence between submodules of $M$ which contains $M'$ and submodules of $M/M'$.
\end{proposition}

\begin{definition}
	[kernel, image] If $f:M\to N$ is an A-module homomorphism, define 
	\[
		\ker(f) = \{x\in M: f(x) = 0\in N\}\\
		Im(f) = f(M)\\
		Coker (f) = N/Im(f)\]
\end{definition}
\begin{proposition}
	$\ker f$ is a submodule of $M$, $Im(f)$ is a submodule of $N$.
\end{proposition}
\begin{proposition}
	\[
		M/\ker f \cong Im(f)\]
\end{proposition}

\section{operations on submodules}
Let $(M_i)_{i\in I}$ be a family of submodules of $M$.
\begin{definition}
	[sum] $\sum M_i$ is the set of all finite sums $\sum x_i$, where $x_i \in M_i$. It's the smallest submodule that contains all the $M_i$.
\end{definition}

\begin{definition}
	[intersection] The intersection $\cap M_i$ is again a submodule of $M$. 
\end{definition}

\begin{definition}
	Let $\alpha$ be an ideal of $A$, $M: A-module$ then $\alpha M$ is a sub A-module. 
\end{definition}

\begin{definition}
	[annihilator]
\end{definition}

\section{direct sum and product}

\section{finitely generated modules}

\begin{definition}
	[free module] A {\it free} $\mathrm{A}-module$ is one which is isomorphic to some $\oplus_{i\in I}A$.
	A finitely generated free $A-module$ is isomorphic to $A\oplus\cdots\oplus A$ ($n$ summands), which is denoted by $A^n$.
\end{definition}
\begin{proposition}
	$M$ is a finitely generated A-module $\iff$ $M$ is isomorphic to a quotient of $A^n$ for some integer $n > 0$.
\end{proposition}
\begin{proof}
	$\implies$ : consider \[\begin{tikzcd}
		{\phi:} & {A^n} && M \\
		& {(a_i)} && {\sum a_ix_i}
		\arrow[from=1-2, to=1-4]
		\arrow[maps to, from=2-2, to=2-4]
	\end{tikzcd}\]
	Then $M\cong A/\ker \phi$.\\
	$\impliedby$: consider 
	\[\begin{tikzcd}
		{A^n} && M \\
		& {A^n/N}
		\arrow["\phi"'{pos=0.6}, color={rgb,255:red,214;green,92;blue,92}, curve={height=6pt}, hook, two heads, from=2-2, to=1-3]
		\arrow["q"', two heads, from=1-1, to=2-2]
		\arrow["f", from=1-1, to=1-3]
	\end{tikzcd}\]
	Then $f:=q\circ \phi$ is surjective homomorphism. And $(f(e_i))$ generates $M$ since $(e_i)$ generates $A^n$, where $e_i = (0, 0, ..., 1, ...,0)$.
\end{proof}

\begin{proposition}
	Let $M$ be a finitely generated A-module, let $\alpha$ be an ideal of $A$, $\phi \in E(M)$ such that $\phi(M) \subseteq \alpha M$. Then there exists $(a_i)$ such that 
	\[\phi^n + a_1\phi^{n-1}+...+a_n = 0\]
\end{proposition}
\begin{proof}
	Let $x_1,...,x_n$ be generators of $M$, then each $\phi(x_i) \in \alpha M$, so that $\phi(x_i) = \sum_{j}a_{ij}x_j$, where $a_{ij} \in \alpha$. So $\phi$ is equivalent to a matrix over ring $\alpha$.
	But $Cayley-Hamilton$ theorem, we know $Char_{\phi}(\phi) = 0 \in E(M)$.  参考高代的证明。
\end{proof}

\begin{corollary}
	Let $M$ be a finitely generated $A-module$, $\alpha$ be an ideal of $A$ such that $\alpha M=M$. Then there exists $x \equiv 1 (mod \ \alpha)$ such that $xM = 0$.
\end{corollary}
\begin{proof}
	take $\phi = id$ in Proposition 3.5.2, and $x = 1 + a_1 + ... + a_n$.
\end{proof}

\begin{theorem}
	[Nakayama's lemma] Let $M$ be a finitely generated $A-module$ and $\alpha$ an ideal of $A$ contained in the Jacobson radical $\mathfrak{R}$ of $A$. Then $\alpha M = M$ implies $M = 0$.
\end{theorem}
\begin{proof}
	By 3.5.3 we get some $x \equiv 1$ (mod $\mathfrak R$), and $x$ is a unit in $A$. Hence $M = 0$.
\end{proof}

% file ending

\chapterimage{./snow_resized.png}
\chapter{Introduction to Homology}

\begin{definition}
	[A-module category] Suppose $A$ is a commutative ring with an identity, we denoted as $Mod_A$ as the class of all $A-modules$.
\end{definition}

\begin{definition}
	[complex 复形, exact 正和] Suppose $M$ is a sequence of $A-modules$ and $A-$homomorphisms
	\[\begin{tikzcd}
		{...} && {M_{i-1}} && {M_i} && {M_{i+1}} && {...}
		\arrow[from=1-1, to=1-3]
		\arrow["{f_i}", from=1-3, to=1-5]
		\arrow["{f_{i+1}}", from=1-5, to=1-7]
		\arrow[from=1-7, to=1-9]
	\end{tikzcd}\]
	is said to be a {\it complex} if for any $i, f_{i+1}\circ f_i = 0$, i.e. $Im(f_i)\subseteq \ker f_{i+1}$. 

	we say it's {\it exact at $M_i$}, if $Im (f_i) = \ker f_{i+1}$
	
	we say it is {\it exact}, if it's exact at all $M_i$.
\end{definition}

\begin{definition}
	[homology group] For a complex (chain) $M$, since $Im(f_i)$ is a submodule of $\ker f_{i+1}$, we define a group 
	\[H_i(M) = \frac {\ker f_{i+1}} {Im (f_i)}\]
	called {\it (co)homology as i}
\end{definition}

\begin{example}
	If $0 \xrightarrow{} M' \xrightarrow{f} M \xrightarrow{g} M^{''}\xrightarrow{}0$ is exact, then 
	\begin{itemize}
		\item $f$ is injective.
		\item $g$ is surjective
		\item $\ker g = Im (f)$
	\end{itemize}
	i.e. $M / Im (f) \cong M''$.
\end{example}

\begin{proposition}
	[exact test] In the category of $Mod_A$, we have 
	\begin{enumerate}
		\item $M'\xrightarrow{u} M \xrightarrow{v} M''\xrightarrow{} 0$ is exact if and only if \\
			  For any $N \in Mod_A$, the sequence $0\xrightarrow{}Hom_A(M^{''},N)\xrightarrow{\overline{v}}Hom_A(M,N)\xrightarrow{\overline{u}}Hom_A(M',N)$ is exact.
		\item $0 \to N' \xrightarrow{u} N \xrightarrow{v} N''$ is exact if and only if \\
			  For any $M \in Mod_A$, the sequence $0 \to Hom(M,N') \xrightarrow{\overline{u}}Hom(M,N)\xrightarrow{\overline{v}}Hom(M,N'')$ is exact.
	\end{enumerate}
\end{proposition}
\begin{proof}
	For (1). \\
	$\implies$ The condition means $v:surj \wedge Im(u) = \ker v$. So $v\circ u = 0 \in Hom(M', M'')$. For any $f \in Hom(M'', N)$, $\overline{v}$ maps $f$ to $f \circ v$. Since $v$ is surjective, then $f_1 \circ v = f_2 \circ v \implies f_1 = f_2$. Thus, $\overline{v}$ is injective. It's easy to show $Im(\overline{v}) \subseteq \ker \overline{u}$. $\ker \overline{u} \subseteq Im(\overline{v})$ can be deduced by the following claim:
	\[\begin{tikzcd}
		M && {M''} && N \\
		m && {v(m)} && {g(m)}
		\arrow["g", curve={height=-30pt}, from=1-1, to=1-5]
		\arrow["v"', two heads, from=1-1, to=1-3]
		\arrow["f"', color={rgb,255:red,214;green,92;blue,214}, from=1-3, to=1-5]
		\arrow[maps to, from=2-3, to=2-5]
		\arrow[maps to, from=2-1, to=2-3]
	\end{tikzcd}\]
	If $g:M\to N$ such that (for any $v(m)=0\in M''$, then $g(m)=0\in N$), then there exists unique $f: M'' \to N$ such that $g = f\circ v$, where all the maps are homomorphisms.
	
	{\it proof of the claim} 
	
	Since $v$ is surjective, the only possible $f$ maps $v(m)$ to $g(m)$. We need to verify $f$ is a module homomorphism. First, it is well defined since $\ker v \subseteq \ker g$. The homomorphism follows naturally.

	$\impliedby$ First, we show $v$ surjective $\iff M''/Im(v) = \{0\}$. Take $N = M''/Im(v)$, we have 
	\[\begin{tikzcd}
		{Hom_A(M'',N)} && {Hom_A(M,N)} \\
		{(M''\xrightarrow{quotient}N)} && M & {M''} & {N = M''/Im(v)}
		\arrow[hook, from=1-1, to=1-3]
		\arrow["v"', from=2-3, to=2-4]
		\arrow["quotient"', from=2-4, to=2-5]
		\arrow["{zero-map}", curve={height=30pt}, from=2-3, to=2-5]
		\arrow[maps to, from=2-1, to=2-3]
	\end{tikzcd}\]
	Hence, the quotient map (a surjective one) is zero map, which means $N = \{0\}$.

	Second, we show $Im(u) \subseteq \ker v$.
	The condition ($\overline{u}\circ\overline{v}$) means the following diagram commutes for any $N\in Mod_A, f\in Hom_A(M'',N)$.
	\[\begin{tikzcd}
		{M'} && M && {M''} \\
		\\
		&&&& N
		\arrow["u"', from=1-1, to=1-3]
		\arrow["v"', from=1-3, to=1-5]
		\arrow["f"', from=1-5, to=3-5]
		\arrow["{zero-map}", from=1-1, to=3-5]
	\end{tikzcd}\]
	Take $N = M'', f = id$, which yields $v\circ u = 0$.
	
	Finally, we show $\ker v \subseteq Im(u)$.
	\[\begin{tikzcd}
		{M'} && { M} && {N=M/Im(u)} \\
		\\
		&&& {M''}
		\arrow["u"', from=1-1, to=1-3]
		\arrow["{f = quotient}"', from=1-3, to=1-5]
		\arrow["v", two heads, from=1-3, to=3-4]
		\arrow["g", color={rgb,255:red,214;green,92;blue,214}, from=3-4, to=1-5]
		\arrow["{zero-map}", curve={height=-30pt}, from=1-1, to=1-5]
	\end{tikzcd}\]
	The conditions and the proposition can be converted as, for any $f:M \to N$ such that $f\circ u = zero-map$, there exists $g:M'' \to N$, such that $f = g\circ v$. Since we know $Im(u)\subseteq \ker v$, we can take $N = M/Im(u)$ and $f = quotient$.
	If $g$ is well-defined, then $\ker v \subseteq \ker f = Im(u)$.

	(2) {\it to be written}
\end{proof}

\newcommand{\mcf}{\mathcal{F}}

\begin{definition}
	[covariant functor 共变函子] A functor $\mathcal{F}:Mod_A \to Mod_B$ consists of the following data \\
	(1) for any $M\in Mod_A$, give $\mcf M \in Mod_B$. \\
	(2) for any $A-module$ homomorphism $g:M\to N$, give a $B-module$ homomorphism $\mcf (g) : \mcf M \to \mcf N$ such that 
	\begin{itemize}
		\item $\mcf(g\circ h) = \mcf(g)\circ \mcf(h)$.
		\item $\mcf(id_M) = id_{\mcf M}$.
	\end{itemize} 	
	Moreover, if $\mcf: Hom_A(M,N) \to Hom_B(\mcf M,\mcf N)$ is a group homomorphism for all $M, N$, i.e. $\mcf(g+h) = \mcf(g) + \mcf(h)$, we say $\mcf$ is an addictive functor. 
\end{definition}

\begin{definition}
	[contravariant functor 反变函子] A functor $\mathcal{F}:Mod_A^{op} \to Mod_B$ consists of the following data \\
	(1) for any $M\in Mod_A$, give $\mcf M \in Mod_B$. \\
	(2) for any $A-module$ homomorphism $g:M\to N$, give a $B-module$ homomorphism $\mcf (g) : \mcf N \to \mcf M$ such that 
	\begin{itemize}
		\item $\mcf(g\circ h) = \mcf(h)\circ \mcf(g)$.
		\item $\mcf(id_M) = id_{\mcf M}$.
	\end{itemize} 	
\end{definition}

\begin{example}
	Let $M,N \in Mod_A$, define functor 
	\[\begin{tikzcd}
		{Hom_A(M,\cdot):} & {Mod_A} && {Mod_A} \\
		& T && {Hom_A(M,T)} \\
		& {(T_1\xrightarrow{f}T_2)} && {Hom(M,T_1)} && {Hom(M,T_2)} \\
		&&& {(M\to T_1)} && {(M\to T_1 \xrightarrow{f}T_2)}
		\arrow[from=1-2, to=1-4]
		\arrow[maps to, from=2-2, to=2-4]
		\arrow[color={rgb,255:red,92;green,92;blue,214}, from=3-4, to=3-6]
		\arrow[color={rgb,255:red,92;green,92;blue,214}, maps to, from=4-4, to=4-6]
		\arrow[maps to, from=3-2, to=3-4]
	\end{tikzcd}\]
	This functor is addictive and covariant.
\end{example}

\begin{proof}
	check that $\mcf(g\circ h) = \mcf(g)\circ \mcf(h)$. Others are trivial.
\end{proof}

\begin{example}
	In a similar way, we can define $Hom_A(\cdot,N)$. Show that it is a contravariant functor.
\end{example}

\newcommand{\mcg}{\mathcal{G}}
\begin{definition}
	[exact functor 正和函子] Let $\mcf:Mod_A \to Mod_B, \mcg:Mod_A^{op}\to Mod_B$ be addictive. \\
	(1) Say $\mcf$ is {\it left exact} if for any short exact seq $0\to M\to N\to R \to 0$ in $Mod_A$, the sequence $0\to \mcf M\to \mcf N \to \mcf R$ is exact. \\
		Respectively, $\mcg$ is {\it left exact} if for any short exact seq $0\to M\to N\to R \to 0$ in $Mod_A$, the sequence $0\to \mcg R\to \mcg N \to \mcg M$ is exact.
	
	(2) Say $\mcf$ is {\it right exact} if for any short exact seq $0\to M\to N\to R \to 0$ in $Mod_A$, the sequence $\mcf M\to \mcf N \to \mcf R\to 0$ is exact. \\
	Respectively, $\mcg$ is {\it left exact} if for any short exact seq $0\to M\to N\to R \to 0$ in $Mod_A$, the sequence $\mcg R\to \mcg N \to \mcg M\to 0$ is exact.

	(3) Say $\mcf (respectively, \mcg)$ is exact if $\mcf (or, \mcg)$ is both left and right exact. 
\end{definition}

\begin{proposition}
	$Hom_A(M,\cdot)$ and $Hom_A(\cdot, N)$ are left exact functors.	
\end{proposition}
\begin{proof}
	By Proposition 4.0.1.
\end{proof}

\
\end{CJK}
\end{document}
